%!TEX root = ../main.tex
%%%%%%%%%%%%%%%%%%%%%%%%%%%%%%%%%%%%%%%%%%%
%
%LEZIONE 08/03/2017 - SECONDA SETTIMANA (3)
%
%%%%%%%%%%%%%%%%%%%%%%%%%%%%%%%%%%%%%%%%%%%
\chapter{Il crittosistema RSA}
	
	In questo capitolo descriveremo in dettaglio il crittosistema RSA.
	Nella parte iniziale richiameremo e svilupperemo alcuni concetti matematici e computazionale che ci saranno utili in seguito.

\section{Introduzione}

	In questo paragrafo richiameremo alcuni fatti di teoria modulare e analizzeremo l'algoritmo di esponenziazione modulare.

	\begin{teor}{di Eulero-Fermat}{teoremaEuleroFermat}
	Siano \(a\in \Z\) e \(n\in\N\) tali che \((a,n)=1\). Allora
		\[
		a^{\j(n)} \equiv 1 \pmod{n}.
		\]
	\end{teor}

	\begin{oss}
	Un caso particolare si ha quando \(n=p\) primo. In tal caso se \(p\nmid a\) si ha \((a,p)=1\) e in particolare
		\[
		a^{\j(p)} = a^{p-1} \equiv 1 \pmod{p} \iff a^p \equiv a \pmod{p}.
		\]
	\end{oss}

	\begin{notz}
	\(\j(n)\) è la funzione di Eulero calcolata in \(n\), ovvero il numero di interi minori di \(n\) e coprimi con esso. Ricordiamo che in generale se \(n=p_1^{\a_1} \cdot\ldots\cdot p_s^{\a_s}\) avremo
		\[
		\j(n) = \j(p_1^{\a_1}) \cdot\ldots\cdot \j(p_s^{\a_s}) = p_1^{\a_1-1}(p_1-1) \cdot\ldots\cdot p_s^{\a_s-1}(p_s-1).
		\]
	Per cui è possibile calcolare \(\j(n)\) se si conosce la fattorizzazione di \(n\).
	\end{notz}

	\begin{cor}\label{cor:teoremaFermatRSA}
	Sia \(n=p\,q\) con \(p,q\) primi. Sia \(m\equiv 1 \pmod{\j(n)}\). Allora
		\[
		a^m \equiv a \pmod{n},\,\fa a \in \Z.
		\]
	\end{cor}

	\begin{proof}
	La tesi è banalmente vera per Eulero se \((a,n)=1\). Supponiamo quindi che \((a,n)>1\).
	Passando in modulo \(p\), vi sono due possibilità:
	\begin{itemize}
		\item Se \(p\mid a\) allora
			\[
			a^m \equiv a \equiv 0 \pmod{p}.
			\]
		\item Se \(p\nmid a\), per ipotesi abbiamo \(m\equiv 1 \pmod{\j(n)}\) dove \(\j(n)=(p-1)(q-1)\), quindi
			\[
			a^m = a^{1+k(p-1)(q-1)} = a\,(a^{p-1})^{k\,(q-1)} \equiv a \pmod{p}.
			\]
	\end{itemize}
	Analogamente si mostra \(a^m \equiv a \pmod{q}\), da cui
		\[
		\begin{cases}
		a^m \equiv a \pmod{p}\\
		a^m \equiv a \pmod{q}
		\end{cases}
		\implies a^m \equiv a \pmod{n},
		\]
	per il teorema cinese dei resti.
	\end{proof}

	In generale, se vogliamo calcolare \(a^n\), dobbiamo eseguire \(n-1\) moltiplicazioni.
	Se \(n\approx 2^k\) questo significa che la complessità dell'esponenziazione è esponenziale.
	Per ridurre tale complessità, l'idea è scrivere \(n\) come somma di potenze di 2 e spezzare la potenza.
	Supponiamo di avere
		\[
		n = 2^k + 2^{k-i} + \ldots + 2^{k-m} \implies a^k = a^{2^k}a^{2^{k-i}} \cdot\ldots\cdot a^{2^{k-m}}.
		\]
	A questo punto è sufficiente calcolare le potenze di \(a\) fino a \(a^{2^k}\), nel processo calcoleremo incidentalmente anche tutte quelle precedenti. Infine basta moltiplicare fra di loro le potenze trovate.
	Osserviamo in particolare che per calcolare ogni potenza è sufficiente una singola moltiplicazione, infatti
		\begin{align*}
		a^2 & = a \cdot a;\\
		a^{2^2} & = a^2 \cdot a^2;\\
		& \ldots\\
		a^{2^k} & = a^{2^{k-1}} \cdot a^{2^{k-1}}.
		\end{align*}

	A noi interessa valutare la complessità di questo algoritmo nel caso modulare

	\begin{prop}{Complessità dell'algoritmo square and multiply}{complessitàSquareMultiply}
	Siano \(b,m,n\in \Z\) con \(L(m) = k\) e \(L(n) = h\).
	Consideriamo l'algoritmo \(A\) per il calcolo di \(b^m \pmod{n}\) tramite la strategia precedente, allora
		\[
		T(A) \in \bO\big((k-1)h^2\big).
		\]
	\end{prop}

	\begin{proof}
	Supponiamo di avere
		\[
		m = a_0 + a_1 2 + a_2 2^2 + \ldots + a_{k-1}2^{k-1} \qquad\text{con }a_i \in \{0,1\}.
		\]
	Pertanto avremo
		\[
		b^m = b^{a_0+a_1 2 + \ldots + a_{k-1}2^{k-1}} = b^{a_0}b^{a_1 2} \cdot\ldots\cdot b^{a_{k-1}2^{k-1}}.
		\]
	Dobbiamo calcolare le \(k\) potenze di \(2\). Come osservato in precedenza il calcolo di \(2^i\) è costituito da una singola moltiplicazione di numeri minori di \(n\). Nel caso computazionalmente peggiore, ovvero se \(a_i=1\) per ogni \(i\), avremo \(k-1\) passi.
	Quindi, ricordando la complessità della moltiplicazione, avremo
		\[
		T(A) \in \bO\big((k-1)h^2\big).
		\]
	\end{proof}

	\begin{oss}
	Se si ha \((b,n)=1\), possiamo supporre che \(m \le \j(n)\). Infatti se così non fosse, potrei scrivere
		\[
		b^m \equiv b^{m'} \pmod{n} \qquad\text{con } m \equiv m' \pmod{\j(n)}.
		\]
	Pertanto se \((b,n)=1\) avremo \(m \le \j(n) \le n\), ovvero \(L(m) \le L(n)\).
	Quindi la complessità dell'esponenziazione può essere stimata completamente in termini di \(h=L(n)\), ottenendo
		\[
		T(A) \in \bO(h^3).
		\]
	\end{oss}
%%%%%%%%%%%%%%%%%%%%%%%%%%%%%%%%%%%%%%%%%
%
%LEZIONE 14/03/2017 - TERZA SETTIMANA (1)
%
%%%%%%%%%%%%%%%%%%%%%%%%%%%%%%%%%%%%%%%%%
\section{Descrizione del crittosistema}

	Come ogni crittosistema a chiave pubblica, anche RSA si basa su una funzione unidirezionale speciale.
	L'idea del crittosistema si basa sulla difficoltà nella fattorizzazione dei numeri.
	In particolare abbiamo visto come moltiplicare due interi a \(n\) bit sia dell'ordine di \(\bO(n^2)\), mentre è possibile dimostrare che fattorizzare un numero a \(n\) bit è un'operazione dell'ordine di \(\bO(2^{c\,n^{1/3}})\).

	Il crittosistema RSA si schematizza come segue:
	\begin{itemize}
		\item Sia \(N=p\,q\) con \(p,q\) primi e siano \(P=C=\Z_N\).
		\item Lo spazio delle chiavi è
			\[
			K = \Set{(N,p,q,d,e) | d\,e \equiv 1 \pmod{\j(N)}}.
			\]
	\end{itemize}

	Da questo insieme avremo
	\begin{itemize}
		\item \((p,q,d)\) la chiave privata.
		\item \((N,e)\) la chiave pubblica.
	\end{itemize}

	Se \(k=(N,p,q,d,e)\) è una chiave, il messaggio \(x\in P\) verrà criptato attraverso
		\[
		e_k(x) = x^e \pmod{N}.
		\]
	Viceversa, un codice \(y\in C\) sarà decriptato con
		\[
		d_k(y) = y^d \pmod{N}.
		\]
	Dalla costruzione della chiave, sappiamo che \(e\,d \equiv 1 \pmod{\j(N)}\), inoltre \(N=p\,q\) con \(p,q\) primi.
	Quindi per il \hyperref[cor:teoremaFermatRSA]{corollario di Fermat} sappiamo che
		\[
		(x^e)^d \equiv x \pmod{N} \iff d_k(e_k(x)) = x.
		\]

	\begin{ese}
	Supponiamo che Bob scelga \(p=17\) e \(q=11\), costruisce quindi
		\[
		N = 17\cdot 11 = 187 \qquad\text{e}\qquad \j(N) = 16\cdot 10 = 160.
		\]
	Bob scegli inoltre \(e=7\) e calcola \(d=e^{-1}=23 \pmod{160}\).
	Pubblica quindi la chiave pubblica \((187,7)\).

	Supponiamo che Alice voglia cifrare il testo in chiaro \(88\), il cifrato sarà quindi
		\[
		88^7 = 11 \pmod{187}.
		\]
	Bob riceve \(11\) da Alice, per decifrare calcola
		\[
		11^{23} = 88 \pmod{187}
		\]
	ritrovando quindi il messaggio originale.
	\end{ese}
	\noindent
	Alla base dell'RSA vi sono quindi una serie di problemi "facili" in fase di cifratura e "difficili" in decifratura.
	Nei prossimi paragrafi ci occuperemo proprio di studiare algoritmi polinomiali per i problemi di cifratura e di dimostrare che tutti i problemi di decifratura sono equivalenti alla fattorizzazione di un intero.

	In particolare i problemi che coinvolgono la cifratura sono
	\begin{itemize}
		\item Determinare se un intero \(n\) è primo.
		\item Dati \(a\) e \(n\), trovare \((e,n)\) e, nel caso \((e,n)=1\), calcolare l'inverso di \(e\) modulo \(n\).
		\item Calcolare la funzione \(x^e \pmod{n}\).
	\end{itemize}
	Degli ultimi due problemi abbiamo già esibito un algoritmo polinomiale, rispettivamente l'algoritmo di Euclide e l'algoritmo square and multiply.
	Ai cosiddetti test di primalità, dedicheremo il prossimo paragrafo.

	Viceversa, i problemi che coinvolgono la decifratura sono
	\begin{itemize}
		\item Fattorizzare un intero \(n\).
		\item Dato un intero \(n\), calcolare \(\j(n)\).
		\item Dati \(n,e\), trovare \(d\) tale che
			\[
			(x^e)^d \equiv x \pmod{n}.
			\]
	\end{itemize}

	A prima vista questi problemi sono esposti in ordine di difficoltà. \'E chiaro infatti che se si riesce a fattorizzare \(n\), gli altri due problemi diventano banali.
	Questo ci fa capire che violare l'RSA non può essere più difficile che fattorizzare un intero.
	D'altronde, vedremo in seguito che questi tre problemi sono tra loro equivalenti.

\section{Test di primalità}

	In questo paragrafo ci occuperemo di descrivere alcuni algoritmi polinomiali per determinare se un interno \(n\) sia primo.
	Come è facile intuire, affinché questi algoritmi siano polinomiali, essi non forniranno alcuna informazione sulla fattorizzazione di \(n\).

	Nella prima parte faremo cenno ad alcuni elementi di teoria dei numeri, per una trattazione più approfondita si consiglia di fare riferimento ad un testo specializzato.

	\begin{teor}{Cardinalità numeri primi}{cardinalitàNumeriPrimi}
	Vi sono infiniti numeri primi.
	\end{teor}

	\begin{proof}
	Supponiamo per assurdo che vi siano un numero finito di primi \(p_1,\ldots,p_k\).
	Definiamo
		\[
		N = p_1 \cdot\ldots\cdot p_k +1.
		\]
	Per costruzione \(N\) non è divisibile per nessuno dei \(p_1,\ldots,p_k\).
	Per cui vi sono altri primi che fattorizzano \(N\), da cui l'assurdo.
	\end{proof}

	\begin{teor}{dei numeri primi}{teoremaNumeriPrimi}
	Sia \(\p(x)\) la funzione aritmetica che enumera i numeri primi fino ad \(x\). Allora
		\[
		\lim_{x \to \infty} \frac{\p(x)}{\frac{x}{\ln x}} = 1
		\]
	\end{teor}

	\begin{oss}
	Tramite il teorema dei numeri primi, è possibile stimare la probabilità che un numero \(N\) sia primo. Tale probabilità è infatti circa
		\[
		\p(N)/N.
		\]
	Ad esempio, la probabilità che un numero casuale con al più cento cifre decimali sia primo è circa
		\[
		\frac{1}{\ln(10^{100})} = \frac{1}{230}.
		\]
	Questa probabilità può essere raffinata, scartando ad esempio i numeri pari e i multipli di tre, scarteremmo \(\frac{2}{3} = \frac{1}{2}+\frac{1}{6}\). La probabilità di trovare un primo con al più cento cifre diventa quindi \(\frac{1}{77}\).

	Questo ci dice che trovare numeri primi di grandi dimensioni è relativamente facile.
	\end{oss}

	\begin{defn}{Numero di Fermat}{numeroFermat}\index{Numero!di Fermat}
	Un numero si dice \emph{di Fermat} se è della forma
		\[
		F_n = 2^{2^n}+1.
		\]
	\end{defn}

	\begin{ese}
	\(F_0=3,F_1=5,F_2=17\) sono numeri di Fermat.
	\end{ese}

	\begin{oss}
	Fermat congetturò che \(F_n\) è primo per ogni \(n\).
	Ciò venne velocemente dimostrato falso fattorizzando \(F_5\).
	\end{oss}

	\begin{prop}{Espressione ricorsiva dei numeri di Fermat}{espressioneRicorsivaNumeriFermat}
	Siano \(F_i\) i numeri di Fermat, allora
		\[
		F_n-2 = \prod_{i=0}^{n-1} F_i.
		\]
	\end{prop}

	\begin{proof}
	Mostriamolo per induzione:
	\begin{itemize}
		\item Per \(n=1\) è banalmente vero, infatti
			\[
			F_1 -2 = 5-2 = 3 = F_0.
			\]
		\item Supponiamo che la tesi sia vera per \(n-1\) e mostriamola per \(n\):
			\[
			\begin{split}
			F_n-2 & = 2^{2^n}+1-2 = 2^{2^n}-1 = \big(2^{2^{n-1}}\big)^2-1 = \big(2^{2^{n-1}}-1\big)\big(2^{2^{n-1}}+1\big)\\
			& = \big(2^{2^{n-1}}+1-2\big)\big(2^{2^{n-1}}+1\big) = (F_{n-1}-2)\,F_{n-1}.
			\end{split}
			\]
		Ma per ipotesi induttiva
			\[
			F_{n-1}-2 = \prod_{i=0}^{n-2}F_i,
			\]
		da cui
			\[
			F_n -2 = \left( \prod_{i=0}^{n-2}F_i \right)\,F_{n-1} = \prod_{i=0}^{n-1}F_i.\qedhere
			\]
	\end{itemize}
	\end{proof}

	\begin{cor}
	Siano \(F_n,F_m\) due numeri di Fermat con \(n\neq m\). Allora
		\[
		(F_n,F_m) = 1.
		\]
	\end{cor}

	\begin{proof}
	Supponiamo per assurdo che \(p\) sia un primo che divide sia \(F_n\) che \(F_m\), dove assumiamo \(n<m\) senza perdita di generalità.
	Dal momento che \(p\mid F_n\) avremo
		\[
		p \mid F_0 \cdot\ldots\cdot F_n \cdot\ldots\cdot F_{m-1}.
		\]
	Inoltre, dalla proposizione precedente,
		\[
		p \mid F_m = F_0 \cdot\ldots\cdot F_{m-1} +2.
		\]
	Quindi \(p\) deve dividere la differenza fra \(F_0 \cdot\ldots\cdot F_{m-1}+2\) e \(F_0 \cdot\ldots\cdot F_{m-1}\) che è \(2\).
	Da cui \(p=2\) che è assurdo in quanto i numeri di Fermat sono per costruzione dispari.
.	\end{proof}

	\begin{oss}
	Da questo corollario segue una dimostrazione alternativa sull'infinità dei numeri primi: infatti, dal momento che esistono infiniti numeri di Fermat e che ogni numero primo ne divide al più uno solo, devono esistere infiniti numeri primi.
	\end{oss}

	\begin{defn}{Test deterministico}{testDeterministico}\index{Test!deterministico}
	Un test si dice \emph{deterministico} se risponde in modo univoco ad un problema, senza margine di errore.
	\end{defn}

	\begin{defn}{Test probabilistico}{testProbabilistico}\index{Test!probabilistico}
	Un test si dice \emph{probabilistico} se consiste di una successione di test \(\{T_m\}_{m\in\N}\) e una successione che tende a zero \(\{\e_m\}_{m\in\N}\) tale che:
	\begin{itemize}
		\item Se il problema non passa un test \(T_m\) allora la risposta è certamente negativa.
		\item La probabilità che il problema superi di test \(T_1, \ldots, T_m\) ma che la risposta sia negativa è minore di \(\e_m\).
	\end{itemize}
	\end{defn}

	\begin{oss}
	Fino al 2002 gli unici test di primalità polinomiali erano probabilistici.
	Nel 2002, venne trovato un algoritmo polinomiale per determinare la primalità con una complessità computazionale di \(\bO(n^{12})\) in seguito migliorata a \(\bO(n^6)\).
	Da ciò sappiamo che il problema di primalità è un problema \(P\).
	\end{oss}

	Dal punto di vista dell'efficienza si preferisce usare un test di primalità probabilistico, in quanto più veloce e sufficientemente affidabile.
	L'idea di un test di primalità probabilistico è quello di cercare prove del fatto che \(n\) si comporti o meno come un primo senza cercare i suoi fattori.
	Si utilizzano quindi condizioni necessarie dei numeri primi, che in caso di fallimento ci garantiscono la non primalità di \(n\), mentre in caso di successo aumentano la probabilità che \(n\) sia primo.

\subsection{Test di Fermat}

	Il test di Fermat è un test di primalità probabilistico che sfrutta il \hyperref[th:teoremaEuleroFermat]{teorema di Fermat}. Ricordiamo che quest'ultimo afferma che, se \(p\) è primo e \(1\le a <p\), allora
		\[
		a^{p-1} \equiv  1 \pmod{p}.
		\]
	Sia \(n\) è l'intero di cui vogliamo determinare la primalità, se troviamo \(a < n\) tale che
		\[
		a^{n-1} \not\equiv 1 \pmod{n},
		\]
	allora sappiamo per certo che \(n\) non è primo pur non avendo studiato i suoi fattori.

	Il test di Fermat per \(n\in\N\) è quindi il seguente:
	prendiamo \(a<n\) e calcoliamo \((a,n)\),
	\begin{itemize}
		\item Se \((a,n)\neq 1\) allora \(n\) non è primo;
		\item Se \((a,n)=1\) calcoliamo \(a^{n-1}\) modulo \(n\). Se è diverso da \(1\) allora \(n\) non è primo. 
	\end{itemize}

	\begin{ese}
	Valutiamo la primalità di \(323\):
		\[
		2^{322} \equiv 157 \pmod{323},
		\]
	quindi \(323\) non è primo.
	\end{ese}
%%%%%%%%%%%%%%%%%%%%%%%%%%%%%%%%%%%%%%%%%
%
%LEZIONE 16/03/2017 - TERZA SETTIMANA (2)
%
%%%%%%%%%%%%%%%%%%%%%%%%%%%%%%%%%%%%%%%%%
	\begin{defn}{Pseudoprimo}{pseudoprimo}\index{Pseudoprimo}
	Si dice che \(n\) è uno \emph{pseudoprimo in base \(a\)}, se \(n\) non è primo ma
		\[
		a^n \equiv a \pmod{n}.
		\]
	\end{defn}

	\begin{oss}
	Il nome pseudoprimo si usa in quanto il teorema di Fermat non fornisce una condizione sufficiente di primalità.
	\end{oss}

	\begin{ese}
	341 è uno pseudoprimo in base 2, infatti
		\[
		2^{340} \equiv 1 \pmod{341},
		\]
	ma \(341=11\cdot 31\).
	\end{ese}

	\begin{defn}{Numero di Carmichael}{numeroCharmichael}\index{Numero!di Carmichael}
	Un intero \(n\) non primo si dice \emph{numero di Carmichael} se è uno pseudoprimo in base \(a\) per ogni
		\[
		1 < a < n \qquad\text{tale che } (a,n) = 1.
		\]
	\end{defn}

	\begin{ese}
	561 è il più piccolo numero di Carmichael.
	\end{ese}

	\begin{oss}
	L'esistenza di tali numeri ci dice che il test di Fermat non è affidabile come test di primalità.
	Ad ogni modo esso fornisce l'idea generale dei test di primalità probabilistici.
	\end{oss}

	\begin{teor}{Cardinalità dei numeri di Carmichael}{cardinalitàNumeriCarmichael}
	Ci sono infiniti numeri di Carmichael.
	\end{teor}

	\begin{proof}
	Non fornita.
	\end{proof}

	\begin{pr}\label{pr:Carmichael1}
	Sia \(n\) un numero di Carmichael.
	Allora \(n\) è prodotto di primi distinti, ovvero
		\[
		n = p_1 p_2 \cdot\ldots\cdot p_s \qquad\text{con }p_i \neq p_j\,\fa i\neq j.
		\]
	\end{pr}

	\begin{pr}
	Sia \(n=p_1 \cdot\ldots\cdot p_s\) prodotto di primi distinti.
	Allora \(n\) è un numero di Carmichael se e soltanto se 
		\[
		p-1 \mid n-1 \,\fa p \mid n.
		\]
	\end{pr}

\subsection{Test di Solovay-Strassen}

	Questo test di primalità, nonostante sia stato superato da test più recenti, ha una grande importanza storica in quanto dimostrò la possibilità di utilizzo del crittosistema RSA.

	Per descriverlo richiamiamo di seguito alcuni concetti di teoria dei numeri, in particolare i residui quadratici, il simbolo di Legendre e quello di Jacobi.

	\begin{defn}{Residuo quadratico}{residuoQuadratico}\index{Residuo quadratico}
	Sia \(p\) un primo dispari.
	\(a\in\Z\) si dice \emph{residuo quadratico modulo \(p\)} se la congruenza
		\[
		x^2 \equiv a \pmod{p}
		\]
	ha soluzione.
	\end{defn}

	\begin{ese}
	Gli interi \(1,3,4,5,9\) sono residui quadratici modulo \(11\).
	\end{ese}

	\begin{oss}
	Modulo \(p\), i residui quadratici sono in numero \(\frac{p-1}{2}\). Precisamente sono
		\[
		1^2, 2^2, \ldots, \left( \frac{p-1}{2} \right)^2.
		\]
	Questa osservazione segue da
		\[
		a^2 \equiv b^2 \iff a^2-b^2 \equiv 0 \iff (a-b)(a+b) \equiv 0 \pmod{p},
		\]
	da cui
		\[
		a + b \equiv 0 \pmod{p} \qquad\text{oppure}\qquad a-b \equiv 0 \pmod{p},
		\]
	che ci dice \(a\equiv \pm b \pmod{p}\).
	\end{oss}

	\begin{notz}
	Con \(\Z_p^*:=U(\Z_p)\) indichiamo il gruppo moltiplicativo di \(\Z_p\).
	\end{notz}

	\begin{oss}
	\(\Z_p\) è un campo, pertanto \(\Z_p^*\) è un gruppo moltiplicativo ciclico.
	\end{oss}

	\begin{defn}{Radice primitiva}{radicePrimitiva}\index{Radice primitiva}
	Consideriamo il gruppo moltiplicativo ciclico \(\Z_p^*\).
	Un generatore \(g\) di \(\Z_p^*\) si dice \emph{radice primitiva modulo \(p\)}.
	\end{defn}

	\begin{pr}
	Sia \(g\) una radice primitiva modulo \(p\).
	Allora i residui quadratici modulo \(p\) sono della forma \(g^k\) con \(k\) pari.
	\end{pr}

	\begin{prop}{Criterio di Eulero}{criterioEulero}
	Sia \(p\) un primo dispari e sia \(a\in \Z\) tale che \(p\nmid a\). Allora \(a\) è un residuo quadratico modulo \(p\) se e soltanto se
		\[
		a^{\frac{p-1}{2}} \equiv 1 \pmod{p}.
		\]
	\end{prop}

	\begin{proof}
	Dal teorema di Fermat, sappiamo che
		\[
		a^{p-1} \equiv 1 \pmod{p}.
		\]
	Da ciò segue
		\[
		\left( a^{\frac{p-1}{2}} \right)^2 \equiv 1 \pmod{p} \implies a^{\frac{p-1}{2}} \equiv \pm 1 \pmod{p}.
		\]
	Sia \(g\) una radice primitiva modulo \(p\), pertanto \(a=g^k\) per qualche \(k\). Segue
		\[
		a^{\frac{p-1}{2}} = g^{k\,\frac{p-1}{2}} \equiv 1 \pmod {p} \iff p-1 \mid k\,\frac{p-1}{2} \iff k\text{ pari}.\graffito{\(p-1\) è l'ordine di \(g\)}
		\]
	Come abbiamo visto nella proprietà precedente, questa è una condizione necessaria e sufficiente affinché \(a\) sia un residuo quadratico modulo \(p\).
	\end{proof}

	\begin{oss}
	Viceversa \(a\) non è un residuo quadratico modulo \(p\) se e soltanto se
		\[
		a^{\frac{p-1}{2}} \equiv -1 \pmod{p}.
		\]
	Si mostra in maniera analoga.
	\end{oss}

	\begin{defn}{Simbolo di Legendre}{simboloLegendre}\index{Simbolo!di Legendre}
	Sia \(p\) un primo dispari e sia \(a\in\Z\). Definiamo il \emph{simbolo di Legendre} come segue
		\[
		\lege{a}{p} = 	\begin{cases}
						1 & \text{se \(a\) è un residuo quadratico modulo \(p\)}\\
						0 & \text{se \(p\mid a\)}\\
						-1 & \text{se \(a\) non è un residuo quadratico modulo \(p\)}
						\end{cases}
		\]
	\end{defn}

	\begin{pr}
		\[
		a \equiv b \pmod{p} \implies \lege{a}{p} = \lege{b}{p}.
		\]
	\end{pr}

	\begin{pr}
		\[
		p \nmid a \implies \lege{a^2}{p} = 1.
		\]
	\end{pr}

	\begin{pr}
		\[
		\lege{a}{p} \equiv a^{\frac{p-1}{2}} \pmod{p}.
		\]
	\end{pr}

	\begin{pr}
		\[
		\lege{a\,b}{p} = \lege{a}{p}\lege{b}{p}.
		\]
	\end{pr}

	\begin{pr}
		\[
		\lege{1}{p} = 1.
		\]
	\end{pr}

	\begin{pr}
		\[
		\lege{-1}{p} = (-1)^{\frac{p-1}{2}} = 	\begin{cases}
												1 & p\equiv 1 \pmod{4}\\
												-1 & p\equiv 3 \pmod{4}
												\end{cases}
		\]
	\end{pr}

	\begin{pr}
		\[
		\lege{2}{p} = (-1)^{\frac{p^2-1}{8}} = 	\begin{cases}
												1 & p\equiv \pm 1 \pmod{8}\\
												-1 & p\equiv \pm 3 \pmod{8}
												\end{cases}
		\]
	\end{pr}
%%%%%%%%%%%%%%%%%%%%%%%%%%%%%%%%%%%%%%%%%%
%
%LEZIONE 21/03/2017 - QUARTA SETTIMANA (1)
%
%%%%%%%%%%%%%%%%%%%%%%%%%%%%%%%%%%%%%%%%%%
	\begin{teor}{Legge di reciprocità quadratica}{leggeReciprocitàQuadratica}
	Siano \(p\) e \(q\) primi dispari distinti. Allora
		\[
		\lege{p}{q} = \lege{q}{p} = (-1)^{\frac{p-1}{2}\frac{q-1}{2}}.
		\]
	\end{teor}

	\begin{oss}
	Il teorema può essere letto anche nel modo seguente:
	\begin{itemize}
		\item Se \(p\) o \(q\) sono congrui a \(1\) modulo \(4\), allora \(p\) è un residuo quadratico modulo \(q\) se e soltanto se \(q\) è un residuo quadratico modulo \(p\).
		\item Se \(p\) e \(q\) sono congrui a \(3\) modulo \(4\), allora \(p\) è un residuo quadratico modulo \(q\) se e soltanto se \(q\) non è un residuo quadratico modulo \(p\).
	\end{itemize}
	\end{oss}

	Di seguito descriveremo il simbolo di Jacobi, questo estende il simbolo di Legendre ad \(n\) qualsiasi non necessariamente primo.

	\begin{defn}{Simbolo di Jacobi}{simboloJacobi}\index{Simbolo!di Jacobi}
	Sia \(n=p_1^{\a_1} \cdot\ldots\cdot p_s^{\a_s}\) dispari e sia \(a\in\Z\). Il \emph{simbolo di Jacobi} di è definito come
		\[
		\jac{a}{n} = \lege{a}{p_1}^{\a_1} \cdot\ldots\cdot \lege{a}{p_s}^{\a_s}.
		\]
	\end{defn}

	\begin{oss}
	Il simbolo di Jacobi nasce con l'intento di semplificare il calcolo del simbolo di Legendre. D'altronde non ci fornisce informazioni sul fatto che \(a\) sia o meno un residuo quadratico modulo \(n\).
	\end{oss}

	\begin{ese}
	Calcoliamo il simbolo di Jacobi \(2\) su \(15\):
		\[
		\jac{2}{15} = \lege{2}{3} \lege{2}{5} = (-1)(-1) = 1,
		\]
	ma \(2\) non è un quadrato modulo \(15\).
	\end{ese}
	\noindent
	Si dimostra che molte delle proprietà del simbolo di Legendre valgono ugualmente per il simbolo di Jacobi:

	\begin{pr}
		\[
		a \equiv b \pmod{n} \implies \jac{a}{n} = \jac{b}{n}.
		\]
	\end{pr}

	\begin{pr}
		\[
		(a,n) = 1 \implies \jac{a^2}{n} = 1.
		\]
	\end{pr}

	\begin{pr}
		\[
		\jac{a\,b}{n} = \jac{a}{n}\jac{b}{n}.
		\]
	\end{pr}

	\begin{pr}
		\[
		\jac{1}{n} = 1.
		\]
	\end{pr}

	\begin{pr}
		\[
		\jac{-1}{n} = (-1)^{\frac{n-1}{2}} = 	\begin{cases}
												1 & n\equiv 1 \pmod{4}\\
												-1 & n\equiv 3 \pmod{4}
												\end{cases}
		\]
	\end{pr}

	\begin{pr}
		\[
		\lege{2}{n} = (-1)^{\frac{n^2-1}{8}} = 	\begin{cases}
												1 & n\equiv \pm 1 \pmod{8}\\
												-1 & n\equiv \pm 3 \pmod{8}
												\end{cases}
		\]	
	\end{pr}

	\begin{pr}
	Se \(m,n\in\Z\) sono dispari, allora
		\[
		\jac{n}{m}\jac{m}{n} = (-1)^{\frac{n-1}{2}\frac{m-1}{2}}.
		\]
	\end{pr}

	\begin{prop}{Complessità computazionale del simbolo di Jacobi}{complessitàComputazionaleJacobi}
	Consideriamo l'algoritmo \(A\) per il calcolo del simbolo di Jacobi di \(a\) su \(n\), allora
		\[
		T(A) \in \bO(k^3).
		\]
	\end{prop}

	\begin{proof}
	Sia \(k\) il massimo tra \(L(a)\) e \(L(n)\). Ad ogni passo del calcolo di \(\jac{a}{n}\) dovremo
	\begin{enumerate}
		\item Ridurre il numeratore modulo il denominatore: complessità \(\bO(k^2)\).
		\item Se il numeratore è pari, bisogna estrarre la parte pari e valutarla.
		La valutazione è una divisione per \(8\): complessità \(\bO(k)\).
		\item Scambiare il numeratore col denominatore tramite la legge di reciprocità e successivamente valutare il segno.
		La valutazione è una divisione per \(4\): complessità \(\bO(k)\).
	\end{enumerate}
	L'algoritmo termina in al più \(k\) passaggi, pertanto la complessità totale è
		\[
		\bO(k^3).\qedhere
		\]
	\end{proof}
%%%%%%%%%%%%%%%%%%%%%%%%%%%%%%%%%%%%%%%%%%
%
%LEZIONE 23/03/2017 - QUARTA SETTIMANA (2)
%
%%%%%%%%%%%%%%%%%%%%%%%%%%%%%%%%%%%%%%%%%%
	Ricordiamo, dalle proprietà sul simbolo di Legendre che, se \(p\) è un primo dispari, allora
		\[
		\lege{a}{p} \equiv a^{\frac{p-1}{2}} \pmod{p}.
		\]
	Sia \(n\) un intero di cui vogliamo determinare la primalità, se troviamo \(a<n\) tale che
		\[
		\jac{a}{n} \not\equiv a^{\frac{n-1}{2}} \pmod{n},
		\]
	allora \(n\) non è primo.

	\begin{oss}
	Chiaramente questa non è una condizione sufficiente. Ad esempio \(25\) non è primo, ma
		\[
		\jac{7}{25} = \jac{25}{7} = \jac{4}{7} = \jac{2^2}{7} = 1.
		\]
	\end{oss}

	Il test di Solovay-Strassen per \(n\in\N\) dispari è il seguente:
	prendiamo \(a<n\) e calcoliamo \((a,n)\),
	\begin{itemize}
		\item Se \((a,n)\neq 1\) allora \(n\) non è primo;
		\item Se \((a,n)=1\) calcoliamo \(\jac{a}{n}\) e \(a^{\frac{n-1}{2}}\) modulo \(n\). Se non coincidono allora \(n\) non è primo.
	\end{itemize}

	\begin{defn}{Pseudoprimo di Eulero}{pseudoprimoEulero}\index{Pseudoprimo!di Eulero}
	Un intero \(n\) si dice \emph{pseudoprimo di Eulero in base \(a\)}, se \(n\) non è primo, \((a,n)=1\) e 
		\[
		\jac{a}{n} \equiv a^{\frac{n-1}{2}} \pmod{n}.
		\]
	\end{defn}

	\begin{notz}
	Per dire che \(n\) è uno pseudoprimo di Eulero in base \(a\) scriveremo
		\[
		n\in PSPE(a).
		\]
	Per gli pseudoprimi semplici scriviamo \(n\in PSP(a)\).
	\end{notz}

	\begin{oss}
	Ogni \(n\) dispari composito è uno pseudoprimo di Euelero in base \(\pm 1\).
	\end{oss}

	\begin{prop}{Pseudprimo di Eulero è pseudoprimo}{pseudoprimoEuleroPseudoprimo}
	Sia \(n\) uno pseudoprimo di Eulero in base \(b\), allora \(n\) è uno pseudoprimo in base \(b\).
	\end{prop}

	\begin{proof}
	Per definizione
		\[
		n\in PSPE(b) \iff \jac{b}{n} \equiv b^{\frac{n-1}{2}} \pmod{n}.
		\]
	Ora \((b,n)=1\), quindi
		\[
		\jac{b}{n} = \pm 1 \implies b^{\frac{n-1}{2}} \equiv \pm 1 \pmod{n},
		\]
	ne segue
		\[
		1 \equiv {\big(b^{\frac{n-1}{2}}\big)}^2 \equiv b^{n-1} \pmod{n},
		\]
	ovvero \(n\in PSP(a)\)
	\end{proof}

	\begin{oss}
	Il viceversa è falso. Infatti 
		\[
		91 \in PSP(3) \qquad\text{ma}\qquad 91 \not\in PSPE(3).
		\]
	\end{oss}

	\begin{lem}
	Sia \(n\) dispari non quadrato perfetto.
	Allora esiste \(b<n\) con \((b,n)=1\) tale che
		\[
		\jac{b}{n} = -1.
		\]
	\end{lem}

	\begin{proof}
	Supponiamo che \(n\) sia primo.
	La tesi in questo caso è banale poiché vi sono \(\frac{n-1}{2}\) non residui quadratici modulo \(n\), per ognuno dei quali si ha
		\[
		\jac{b}{n} = -1.
		\]
	Supponiamo quindi che \(n\) sia composito.
	Per ipotesi \(n\) non è un quadrato perfetto, pertanto \(n=p^e s\) con \(p\) primo, \(e\) dispari e \((p,s)=1\).
	Sia \(t\) un non residuo quadratico modulo \(p\).
	Consideriamo il seguente sistema di congruenze:
		\[
		\begin{cases}
		x \equiv t \pmod{p}\\
		x \equiv 1 \pmod{s}
		\end{cases}
		\]
	Dal momento che \((p,s)=1\), tale sistema ha soluzione unica modulo \(p\,s\) per il teorema cinese dei resti. Chiamiamo \(b\) tale soluzione.\graffito{\(b<n\) in quanto \(b\) è modulo \(p\,s\)}
	Mostriamo che \(b\) soddisfa la tesi:
		\[
		\jac{b}{n} = \jac{b}{p}^e \jac{b}{s} = \jac{t}{p}^e \jac{1}{s} = \jac{t}{p}^e = (-1)^e = -1
		\]
	in quanto \(e\) è dispari.
	\end{proof}

	\begin{prop}{Non esistenza di pseudoprimi di Eulero per qualsiasi base}{nonEsistenzaPSPEQualsiasiBase}
	Sia \(n\) dispari composito.
	Allora esiste \(1<b<n\) con \((b,n)=1\) tale che \(n\) non è uno pseudoprimo di Eulero in base \(b\).
	\end{prop}

	\begin{proof}
	Supponiamo per assurdo che per ogni \(b\), con \((b,n)=\) e \(b<n\), si abbia che \(n\) è uno pseudoprimo di Eulero in base \(b\).
	Per la proposizione precedente ciò implica che \(n\) è uno pseudoprimo in base \(b\).
	Ma se ciò accade per ogni possibile base \(b\), allora \(n\) è un numero di Carmichael.
	Sappiamo, da una \hyperref[pr:Carmichael1]{precedente proprietà}, che tali numeri sono prodotti di primi distinti, pertanto
		\[
		n = p_1 \cdot\ldots\cdot p_l.
		\]
	Per il lemma precedente esisterà \(\overline{b}\) tale che
		\[
		\jac{\overline{b}}{n} = -1.
		\]
	Per tale base \(\overline{b}\) avremo, in particolare
		\[
		\jac{\overline{b}}{n} \equiv \overline{b}^{\frac{n-1}{2}} \equiv -1 \pmod{n}.
		\]
	Consideriamo il seguente sistema di congruenze:
		\[
		\begin{cases}
		x \equiv \overline{b} \pmod{p_1}\\
		x \equiv 1 \pmod{p_2 \cdot\ldots\cdot p_l}
		\end{cases}
		\]
	Dal momento che \((p_1, p_2 \cdot\ldots\cdot p_l)=1\), per il teorema cinese dei resti esiste una soluzione \(a\) modulo \(p_1 \cdot\ldots\cdot p_l = n\).
	Mostriamo che \(n\) non è uno pseudoprimo di Eulero modulo \(n\):
	per costruzione
		\[
		a^{\frac{n-1}{2}} \equiv \overline{b}^{\frac{n-1}{2}} \equiv -1 \pmod{p_1},
		\]
	mentre
		\[
		a^{\frac{n-1}{2}} \equiv 1 \pmod{p_2 \cdot\ldots\cdot p_l}.
		\]
	Ne segue che
		\[
		a^{\frac{n-1}{2}} \not\equiv \pm 1 \pmod{n}
		\]
	poiché altrimenti verrebbe violata una delle due congruenze precedenti. Quindi
		\[
		\jac{a}{n} \not\equiv a^{\frac{n-1}{2}} \pmod{n},
		\]
	ovvero \(n\) non è uno pseudoprimo di Eulero in base \(a\). Ciò è assurdo, da cui la tesi.
	\end{proof}

	\begin{oss}
	Questa proposizione ci garantisce che non esistono numeri analoghi a quelli di Carmichael per gli psuedoprimi di Eulero.
	\end{oss}

	\begin{lem}
	Sia \(n\) dispari composito e siano \(b_1,b_2\), con \(1<b_1,b_2<n\) e \((b_1,n)=(b_2,n)=1\), tali che \(n\) è uno pseudoprimo di Eulero in base \(b_1\) e \(b_2\), allora
		\[
		n\in PSPE(b_1 b_2) \qquad\text{e}\qquad n\in PSPE(b_1 b_2^{-1}),
		\]
	con \(b_2^{-1}\) l'inverso di \(b_2\) modulo \(n\).
	\end{lem}

	\begin{prop}{Stima del numero di basi psudoprime di Eulero per un intero}{stimaBasiPseudoprimeEulero}
	Sia \(n\) dispari composito.
	Allora la basi \(b\), con \((b,n)=1\) e \(b<n\), tali che \(n\) è uno pseudoprimo di Eulero in base \(b\), sono non più della metà di tutte le possibili basi \(b\).
	\end{prop}

	\begin{proof}
	Sia \(a<n\) con \((a,n)=1\) tale che \(n\) non è uno pseudoprimo di Eulero modulo \(a\).
	Sia inoltre \(P\) definito come l'insieme
		\[
		P = \Set{b\in U(\Z_n) | n \in PSPE(b)} \subseteq U(\Z_n).
		\]
	Se \(b\in P\), allora \(a\,b\not\in P\), poiché altrimenti si avrebbe \(a \in P\) per il lemma precedente.
	Consideriamo quindi la seguente applicazione
		\[
		f_a \colon P \longrightarrow U(\Z_p) \setminus P, b \longmapsto a\,b.
		\]
	\(f_a\) è banalmente iniettiva, pertanto
		\[
		\abs{P} \le \abs{U(\Z_p) \setminus P},
		\]
	da cui
		\[
		\abs{P} \le \abs{U(\Z_p) \setminus P} \le \abs{U(\Z_p)} - \abs{P} \implies \abs{P} \le \frac{1}{2}\,\abs{U(\Z_p)}.
		\]
	\end{proof}

	\begin{oss}
	Questo ci dice che la probabilità che \(n\) superi un test di Solovay-Strassen senza essere effettivamente primo è minore di \(1/2\).

	Dopo \(n\) iterazioni positive, la probabilità che il numero testato non sia primo è minore di \(1/2^n\).
	\end{oss}
%%%%%%%%%%%%%%%%%%%%%%%%%%%%%%%%%%%%%%%%%%
%
%LEZIONE 28/03/2017 - QUINTA SETTIMANA (1)
%
%%%%%%%%%%%%%%%%%%%%%%%%%%%%%%%%%%%%%%%%%%
	\subsection{Test di Miller-Rabin}

	Sia \(p\) un primo dispari e sia \(a<p\). Per il piccolo teorema di Fermat sappiamo che
		\[
		a^{p-1} \equiv 1 \pmod{p}.
		\]
	Pertanto
		\[
		a^{\frac{p-1}{2}} \equiv \pm 1 \pmod{p}.
		\]
	Se
		\[
		\frac{p-1}{2} \text{ è pari} \qquad\text{e}\qquad a^{\frac{p-1}{2}} \equiv 1 \pmod{p},
		\]
	allora siamo nella stessa situazione precedente e possiamo iterare il procedimento.
	In generale, fintanto che
		\[
		\frac{p-1}{2^k} \text{ è pari} \qquad\text{e}\qquad a^{\frac{p-1}{2^k}} \equiv 1 \pmod{p}
		\]
	è possibile continuare ad estrarre una radice.

	Questo ci dice che se scriviamo \(p-1=2^s t\) con \(t\) dispari e preso \(a<p\) calcoliamo
		\[
		a^t, a^{2t}, a^{2^2 t}, \ldots, a^{2^{s-1}t}, a^{2^s t} \qquad\text{modulo }s
		\]
	certamente \(a^{2^s t} \equiv 1 \pmod{p}\), ma per quanto appena osservato, possiamo trovare un minimo \(k\), con \(0\le k < s\), tale che
		\[
		a^{2^k t} \equiv -1 \pmod{p}.
		\]
	Inoltre, per tutti le esponenziazioni successive a \(k\) la congruenza sarà sempre uguale ad \(1\).

	\begin{ese}
	Consideriamo \(p=13\), pertanto \(p-1=12 = 2^2\cdot 3\). Prendiamo \(a=2\) e procediamo a calcolare gli esponenziali modulari:
		\begin{gather*}
		2^3 \equiv 8 \not\equiv -1 \pmod{13}\\
		2^6 \equiv 64 \equiv -1 \pmod{13}.
		\end{gather*}
	Quindi \(k=1\) e
		\[
		2^{2^h 3} \equiv 1 \pmod{13}\,\fa h > 1.
		\]
	\end{ese}

	\begin{defn}{Pseudoprimo forte}{pseudoprimoForte}\index{Pseudoprimo!forte}
	Sia \(n\) dispari composito. \(n\) si dice \emph{pseudoprimo forte in base \(b\)}, con \(1<b<n\) e \((b,n)=1\), se, scritto \(n-1=2^s t\) con \(t\) dispari, si ha
		\[
		b^t \equiv 1 \pmod{n} \qquad\text{oppure}\qquad \ex 0 \le r < s: b^{2^r t} \equiv -1 \pmod{n}.
		\]
	\end{defn}

	\begin{notz}
	Se \(n\) è uno pseudoprimo forte in base \(b\), scriviamo
		\[
		n \in PSPF(b).
		\]
	\end{notz}

	\begin{ese}
	Mostriamo che \(25 \in PSPF(7)\).
	Iniziamo con lo scrivere \(25-1=2^3\cdot 3\) e procediamo calcolando le esponenziazioni modulari di \(7\):
		\begin{gather*}
		7^3 \equiv 7^2 \cdot 7 \equiv -7 \pmod{25}\\
		7^6 \equiv (-7)^2 \equiv 49 \equiv -1 \pmod{25}.
		\end{gather*}
	\end{ese}

	\begin{prop}{Pseudoprimo forte è pseudoprimo}{pseudoprimoFortePseudoprimo}
	Sia \(n\) dispari composito e sia \(1<b<n\) con \((b,n)=1\). Allora
		\[
		n \in PSPF(b) \implies n \in PSP(b).
		\]
	\end{prop}

	\begin{proof}
	Discende direttamente dalla definizione di pseudoprimo forte.
	\end{proof}

	\begin{prop}{Relazione tra pseudoprimi forti e di Eulero}{relazionePseudoprimiFortiEulero}
	Sia \(n\) dispari composito e sia \(1<b<n\) con \((b,n)=1\).
	\begin{itemize}
		\item Se \(n\equiv 3 \pmod{4}\) allora
			\[
			n \in PSPF(b) \iff n \in PSPE(b).
			\]
		\item Se \(n\equiv 1 \pmod{4}\) allora
			\[
			n \in PSPF(b) \implies n \in PSPE(b).
			\]
	\end{itemize}
	\end{prop}

	\begin{proof}
	Mostriamo solo il primo punto:

	\graffito{\(\Leftarrow)\)}Se \(n\equiv 3 \pmod{4}\) si ha necessariamente \(n-1=2\,t\).
	Pertanto la definizione di pseudoprimo forte in base \(b\) diventa
		\[
		b^t \equiv 1 \pmod{n} \qquad\text{oppure}\qquad b^t \equiv -1 \pmod{n},
		\]
	ovvero
		\[
		b^{\frac{n-1}{2}} \equiv \pm 1 \pmod{n}
		\]
	Inoltre da \(n\equiv 3 \pmod{4}\) segue
		\[
		\jac{1}{n} = 1 \qquad\text{e}\qquad \jac{-1}{n} = -1.
		\]
	Quindi
		\[
		n \in PSPE(b) \implies \pm 1 = \jac{b}{n} \equiv b^{\frac{n-1}{2}} \equiv \pm 1 \pmod{n},
		\]
	che abbiamo visto essere, in questo particolare caso, ad \(n\in PSPF(b)\).

	\graffito{\(\Rightarrow)\)}Supponiamo \(n\in PSPF(b)\).
	Osserviamo che
		\[
		n \equiv 3 \pmod{4} \implies \frac{n-3}{4} \in \Z \implies \jac{b^{2\,\frac{n-3}{4}}}{n} = 1.
		\]
	Pertanto
		\[
		\jac{b}{n} = \jac{b}{n}\jac{b^{2\,\frac{n-3}{4}}}{n} = \jac{b^{\frac{n-3}{2}+1}}{n} = \jac{b^{\frac{n-1}{2}}}{n}.
		\]
	Ora \(n\in PSPF(b)\) e \(n\equiv 3 \pmod{4}\) ci dicono \(b^{\frac{n-1}{2}}\equiv \pm 1 \pmod{n}\) per quanto visto in precedenza. Per cui
		\[
		\jac{b^{\frac{n-1}{2}}}{n} = \jac{\pm 1}{n} = \pm 1 \equiv b^{\frac{n-1}{2}} \pmod{n},
		\]
	quindi
		\[
		\jac{b}{n} \equiv b^{\frac{n-1}{2}} \pmod{n} \implies n \in PSPE(b).\qedhere
		\]
	\end{proof}

	\begin{prop}{Stima del numero di basi pseudoprime forti per un intero}{stimaBasiPseudoprimeForti}
	Sia \(n\) dispari composito. Allora le basi \(b\), con \((b,n)=1\) e \(b<n\), tali che \(n\) è uno pseudoprimo forte in base \(b\), sono non più di un quarto di tutte le possibili basi \(b\).
	\end{prop}

	\begin{proof}
	Non fornita.
	\end{proof}

	Descriviamo di seguito il generico test di Miller-Rabin su un intero \(n\) dispari:
	\begin{enumerate}
		\item Scriviamo \(n-1=2^s t\) con \(t\) dispari.
		\item Scegliamo \(b<n\) e verifichiamo \((b,n)>1\), altrimenti \(n\) non è primo.
		\item Calcoliamo \(b^t\)
		\begin{itemize}
			\item Se \(b^t \equiv \pm 1 \pmod{n}\) allora \(n\) è un possibile primo.
			\item Altrimenti calcolo \(b^{2t}\).
		\end{itemize}
		\item Calcoliamo \(b^{2t}\)
		\begin{itemize}
			\item Se \(b^{2t} \equiv -1 \pmod{n}\) allora \(n\) è un possibile primo.
			\item Se \(b^{2t} \equiv 1 \pmod{n}\) allora \(n\) non è primo.
			\item Altrimenti calcolo \(b^{2^2 t}\).
		\end{itemize}
		\item Itero il procedimento calcolando \(b^{2^k t}\) con \(k \le s-1\) fino a che non ottengo una risposta o fino a che non arrivo a \(b^{2^{s-1}t}\).
		\begin{itemize}
			\item Se \(b^{2^{s-1}t}\equiv -1 \pmod{n}\) allora \(n\) è un possibile primo.
			\item Altrimenti \(n\) non è primo.
		\end{itemize}
	\end{enumerate}

	\begin{oss}
	Dopo \(m\) iterazioni positive del test, posso concludere che \(n\) è un falso primo con probabilità minore di \(\frac{1}{2^{2m}}\).
	\end{oss}
%%%%%%%%%%%%%%%%%%%%%%%%%%%%%%%%%%%%%%%%%%
%
%LEZIONE 30/03/2017 - QUINTA SETTIMANA (2)
%
%%%%%%%%%%%%%%%%%%%%%%%%%%%%%%%%%%%%%%%%%%
\section{Problemi di fattorizzazione}

	Descrivendo RSA abbiamo già osservato come i problemi difficili che rendono sicuro l'algoritmo siano legati alla fattorizzazione di \(N\).
	In particolare dimostreremo come fattorizzare \(N\) sia equivalente a calcolare \(\j(N)\).
	Infine andremo a dimostrare l'equivalenza anche con il terzo problema, ovvero dati \(N\) ed \(e\), trovare \(d\) tale che
		\[
		{(x^e)}^d \equiv x \pmod{N}.
		\]

	\begin{prop}{Equivalenza tra fattorizzazione e calcolo della funzione di Eulero}{equivalenzaFattorizzazioneEulero}
	Sia \(N=p\,q\). Allora conosco la fattorizzazione di \(N\) se e soltanto se posso calcolare \(\j(N)\).
	\end{prop}

	\begin{proof}
	\graffito{\(\Rightarrow)\)} Banale in quanto se conosco \(N=p\,q\) posso calcolare \(\j(N)\) tramite la definizione:
		\[
		\j(N) = \j(p)\j(q) = (p-1)(q-1).
		\]
	\graffito{\(\Leftarrow)\)}Supponiamo di conoscere il valore di \(\j(N)\). Per ipotesi sappiamo che \(N\) è il prodotto di due primi \(p\) e \(q\), per cui
		\[
		\j(N) = (p-1)(q-1) = p\,q - (p+q) +1 = N-(p+q)+1.
		\]
	A questo punto per trovare \(p,q\) possiamo risolvere il sistema
		\[
		\begin{cases}
		p\,q = N\\
		p+q = N-\j(N)+1
		\end{cases}
		\]
	\end{proof}

	\begin{prop}{Congruenza quadratica modulo \(N\)}{congruenzaQuadratica}
	Sia \(N=p\,q\) con \(p,q\) primi. Allora la congruenza \(x^2 \equiv a \pmod{N}\) ha quattro o nessuna soluzione, in particolare
		\[
		\begin{cases}
		4\text{ soluzioni} & \text{se }\jac{a}{p}=1\text{ e }\jac{a}{q}=1\\
		0\text{ soluzioni} & \text{se }\jac{a}{p}=-1\text{ o }\jac{a}{q}=-1\\
		\end{cases}
		\]
	\end{prop}

	\begin{proof}
	Supponiamo che
		\[
		\jac{a}{p}=\jac{a}{q}=1.
		\]
	Allora \(x^2 \equiv a \pmod{p}\) ha due soluzioni \(a_1,a_2\) e \(x^2 \equiv a \pmod{q}\) ha due soluzioni \(b_1,b_2\).
	Le quattro soluzioni modulo \(N\) saranno quindi le soluzioni dei sistemi
		\begin{gather*}
		\begin{cases}
		x \equiv a_1 \pmod{p}\\
		x \equiv b_1 \pmod{q}
		\end{cases}\qquad \qquad
		\begin{cases}
		x \equiv a_1 \pmod{p}\\
		x \equiv b_2 \pmod{q}
		\end{cases}\\
		\begin{cases}
		x \equiv a_2 \pmod{p}\\
		x \equiv b_1 \pmod{q}
		\end{cases}\qquad \qquad
		\begin{cases}
		x \equiv a_2 \pmod{p}\\
		x \equiv b_2 \pmod{q}
		\end{cases}\qedhere
		\end{gather*}
	\end{proof}

	\begin{oss}
	In generale si dimostra che se \(N=p_1^{\a_1} \cdot\ldots\cdot p_s^{\a_s}\), allora la congruenza \(x^2 \equiv a \pmod{N}\) ha \(2^s\) soluzioni se
		\[
		\jac{a}{p_1} = \jac{a}{p_2} = \ldots = \jac{a}{p_s} = 1
		\]
	e zero altrimenti.
	\end{oss}
	\noindent
	Prendiamo sempre \(N=p\,q\) e consideriamo \(x^2\equiv 1 \pmod{N}\). In questo caso avremo sempre
		\[
		\jac{1}{p} = \jac{1}{q} = 1.
		\]
	Per cui vi sono quattro soluzioni modulo \(N\). In particolare ve ne sono due, dette \emph{radici banali}, soluzioni dei sistemi
		\[
		\begin{cases}
		x \equiv 1 \pmod{p}\\
		x \equiv 1 \pmod{q}
		\end{cases} \implies x \equiv 1 \pmod{N};\qquad
		\begin{cases}
		x \equiv -1 \pmod{p}\\
		x \equiv -1 \pmod{q}
		\end{cases} \implies x \equiv -1 \pmod{N}
		\]
	E altre due, le \emph{radici non banali}, che risolvono i sistemi
		\[
		\begin{cases}
		x \equiv 1 \pmod{p}\\
		x \equiv -1 \pmod{q}
		\end{cases}\qquad \qquad
		\begin{cases}
		x \equiv -1 \pmod{p}\\
		x \equiv 1 \pmod{q}
		\end{cases}
		\]

	\begin{ese}
	Sia \(N=403=13\cdot 31\). Le soluzioni di \(x^2 \equiv 1 \pmod{403}\) sono
		\[
		1,92,311,402,
		\]
	dove \(1,402\) sono le radici banali.
	\end{ese}
	\noindent
	Supponiamo di conoscere le radici non banali senza conoscere la fattorizzazione.
	In particolare, se \(x\) è una radice non banale avremo
		\[
		\begin{cases}
		x \equiv 1 \pmod{p}\\
		x \equiv -1 \pmod{q}
		\end{cases}
		\implies p \mid x-1
		\]
	quindi, calcolando \((N,x-1)\) o \((N,x+1)\), otterremmo un fattore di \(N\).

	\begin{oss}
	Quindi in Miller-Rabin, se nel calcolo di
		\[
		a^t, a^{2t}, \ldots, a^{2^k t},
		\]
	incontriamo un \(1\) che non sia in prima posizione e che non sia preceduto da \(-1\), detto \(k\) l'esponente per cui ciò accade, avremo che
		\[
		a^{2^{k-1}t}
		\]
	è una radice non banale dell'unità.
	Pertanto possiamo concludere non solo che \(N\) non è primo, ma ne abbiamo persino trovato un fattore.
	\end{oss}

	Esiste un algoritmo probabilistico che, dati \(N,e,d\), produce radici non banali dell'unità modulo \(N\).
	Tale algoritmo prende in input un \(g\) con \(1<g<N\) e ritorna una radice non banale dell'unità oppure fallisce. Descriviamone il funzionamento:
	\begin{enumerate}
		\item Prendo \(g\in \Z_n\) e controllo che \((g,N)=1\), altrimenti ho fattorizzato \(N\) e l'algoritmo si ferma.
		\item Per ipotesi \(e\,d\equiv 1 \pmod{\j(N)}\), pertanto \(e\,d-1 =k\,\j(N)\). Inoltre \(g\in U(\Z_n)\), quindi applicando Fermat ottengo
			\[
			g^{e\,d-1} = g^{k\,\j(N)} = {(g^{\j(N)})}^k \equiv 1 \pmod{N}
			\]
		\item Scrivo \(e\,d-1=2^s t\) con \(t\) dispari ed eseguo una successione di esponenziazioni modulo \(N\):
			\[
			g^t, g^{2t}, g^{2^2 t}, \ldots, g^{2^s t},
			\]
		dove sappiamo che \(g^{2^s t} \equiv 1 \pmod{N}\).
		\item Se \(g^t \equiv 1 \pmod{N}\) l'algoritmo fallisce perché non posso trovare un ulteriore radice.
		\item Se \(g^{2^l t} \equiv 1\) ma \(g^{2^{l-1}t}\equiv -1\) l'algoritmo fallisce perché ho trovato una radice banale.
		\item Altrimenti esiste \(h\) tale che \(g^{2^h t} \equiv 1 \pmod{N}\) ma \(g^{2^{h-1}t}\not\equiv \pm 1 \pmod{N}\). In tal caso l'algoritmo restituisce
			\[
			g^{2^{h-1}t}
			\]
		come radice non banale dell'unità modulo \(N\).
	\end{enumerate}

	\begin{oss}
	Si dimostra che la probabilità che l'algoritmo fallisca è minore di un mezzo.
	\end{oss}

	\begin{prop}{Equivalenza della fattorizzazione con il terzo problema}{equivalenzaFattorizzazioneTerzoProblema}
	Sia \(N=p\,q\) e sia \(e\) con \(\big(e,\j(N)\big)=1\).
	Allora conosco la fattorizzazione di \(N\) se e soltanto se posso trovare \(d\) tale che
		\[
		{(x^e)}^d \equiv x \pmod{N}.
		\]
	\end{prop}

	\begin{proof}
	\graffito{\(\Rightarrow)\)} Banale, poiché se conosco la fattorizzazione di \(N\) posso calcolare \(\j(N)\). Trovare \(d\) si riduce quindi al calcolo dell'inverso di \(e\) modulo \(\j(N)\).

	\graffito{\(\Leftarrow)\)} Conoscendo \(N,e,d\) possiamo applicare l'algoritmo precedente per trovare una radice non banale dell'unità modulo \(N\). Tramite questa abbiamo visto come sia possibile fattorizzare \(N\).
	\end{proof}

	\begin{ese}
	Siano \(N=403, e=23\) e \(d=47\). Scriviamo
		\[
		e\,d-1 = 1800 = 2^3 \cdot 225.
		\]
	Prendiamo \(g=2\) e cominciamo con il calcolo delle potenze:
		\begin{gather*}
		2^{225} \equiv 187 \pmod{403}\\
		2^{2\cdot 225} \equiv 311 \pmod{403}\\
		2^{4\cdot 225} \equiv 1 \pmod{403}
		\end{gather*}
	quindi \(311\) è una radice non banale dell'unità modulo \(403\).
	Possiamo quindi fattorizzare \(403\) calcolando
		\[
		(403,311-1) = 31 \qquad\text{e}\qquad (403,311+1) = 13.
		\]
	\end{ese}
%%%%%%%%%%%%%%%%%
%ATTACCHI AD RSA%
%%%%%%%%%%%%%%%%%
\section{Attacchi ad RSA}

	In questo paragrafo ci occuperemo di alcuni attacchi ad RSA che non coinvolgono la fattorizzazione del modulo.

	Per prima cosa osserviamo come l'equivalenza fra la fattorizzazione e la conoscenza dell'esponente di decifratura possono esporre il crittosistema ad usi che lo rendono insicuro.
	Nelle applicazioni reali, esiste un superutente, ovvero un'autorità certificata anche detta \emph{Trent}, che distribuisce le chiavi RSA agli utenti che ne fanno richiesta.
	Per risparmiare il Trent potrebbe decidere di fornire a tutti lo stesso modulo \(N\) variando solo gli esponenti di cifratura e decifratura \(e,d\).
	Le varie chiavi sarebbero quindi del tipo
		\[
		(N,e_A,d_A), (N,e_B,d_B), (N,e_C,d_C), \ldots
		\]
	Questo approccio rende del tutto insicuro il sistema, infatti ogni utente possiede \(e,d\) ed è pertanto in grado di fattorizzare \(N\) e violare tutti gli altri utenti con il suo stesso modulo.

	Questo riutilizzo del modulo \(N\) è pericoloso anche nel caso in cui \(A,B\) siano due utenti "amici" che condividono il modulo.
	Supponiamo infatti che \(C\) debba mandare lo stesso messaggio \(m\) ad \(A\) e \(B\). I messaggi cifrati saranno rispettivamente
		\[
		c_A = m^{e_A} \pmod{N} \qquad\text{e}\qquad c_B = m^{e_B} \pmod{N}.
		\]
	Supponiamo che l'attaccante sia a conoscenza di queste informazioni. In tal caso è in grado di decifrare il messaggio se \((e_A,e_B)=1\).\graffito{Ipotesi quasi sempre soddisfatta} Infatti
		\[
		(e_A, e_B) = 1 \implies \,\ex s,t: 1 = s\,e_A + t\,e_B.
		\]
	Ora l'attaccante conosce \(c_A,c_B\), per cui calcola
		\[
		c_A^s \, c_B^t = (m^{e_A})^s (m^{e_B})^t = m^{s\,e_A+t\,e_B} = m.
		\]
	Quindi non bisogna mai utilizzare lo stesso modulo \(N\) per utenti diversi
\subsection{Attacco di Wiener}
	
	L'attacco di Wiener ad RSA è un \emph{attacco ad esponente di decifratura piccolo}.
	Questo tipo di attacco funziona infatti nelle ipotesi in cui
		\[
		\sqrt{6}d < \sqrt[4]{N} \qquad\text{e}\qquad q < p < 2q,
		\]
	dove \(e,d\) sono gli esponenti di cifratura e decifratura e \(N=p\,q\) è il modulo.

	Per descrivere questo tipo di attacco è necessario qualche accenno alla teoria delle frazioni continue

	\begin{defn}{Frazione continua}{frazioneContinua}
	Sia \(x\in \R\), se ne da la sua espressione in \emph{frazione continua} come
		\[
		x = a_0 + \cfrac{1}{a_1 + \cfrac{1}{a_2+\cfrac{1}{a_3+\ldots}}} = [a_0;a_1,a_2,a_3,\ldots]
		\]
	dove gli \(a_i\) sono interi positivi.
	\end{defn}

	\begin{ese}
	Scriviamo in frazione continua \(\frac{214}{35}\). Per farlo applichiamo l'algoritmo di Euclide delle divisioni con il resto:
		\begin{align*}
		214 & = 6\cdot 35 + 4\\
		35 & = 8\cdot 4 + 3\\
		4 & = 1\cdot 3 + 1\\
		3 & = 3\cdot 1 + 0
		\end{align*}
	a questo punto è sufficiente considerare i quozienti parziali per ottenere la scrittura in frazione continua:
		\[
		\frac{214}{35} = 6+\cfrac{1}{8+\cfrac{1}{1+\cfrac{1}{3}}} = [6;8,1,3].
		\]
	\end{ese}

	\begin{pr}
	Sia \(x\in\Q\), allora la scrittura in frazione continua di \(x\) ha un numero finito di termini.
	\end{pr}

	\begin{defn}{Convergenti}{convergenti}
	Consideriamo \(x\in\Q\) nella scrittura in frazione continua \([a_1;a_2,\ldots,a_n]\).
	Se ne definisce il \emph{\(k\)-esimo convergente} \(C_k\) troncando l'espressione in frazione continua al \(k\)-esimo posto, ovvero
		\[
		C_k = [a_1;a_2, \ldots, a_k] \qquad\text{con }k<n.
		\]
	\end{defn}

	\begin{oss}
	Per un generico razionale \(\frac{m}{n}\) si hanno \(k=\max\{L(n),L(m)\}\) convergenti, che è pari al numero di passi nell'algoritmo euclideo.
	\end{oss}

	\begin{oss}
	In generale si dimostra che, preso \(x\in\Q\), si ha
		\[
		C_1 \le C_3 \le \ldots \le C_{2h+1} \le n \le C_{2h} \le \ldots \le C_4 \le C_2.
		\]
	Quindi i convergenti dispari approssimano \(n\) dal basso mentre quelli pari lo approssimano dall'alto. Inoltre, i convergenti sono sempre crescenti.
	\end{oss}


	\begin{ese}
	Nell'esempio precedente \(x=[6;8,1,3]\), da cui
		\[
		C_1 = 6; \qquad C_2 = 6+\frac{1}{8} = \frac{49}{8}; \qquad C_3 = 6+\frac{1}{8+1} = \frac{55}{9}; \qquad C_4 = \frac{214}{35}=x.
		\]
	\end{ese}

	\begin{teor}{Distanza di un convergente dal razionale di partenza}{distanzaConvergente}
	Sia \(x\in\Q\) e siano \(p,q\in\Z\) non necessariamente primi.
	Allora
		\[
		\abs*{x- \frac{p}{q}} < \frac{1}{2q^2} \implies \frac{p}{q} \text{ è un convergente per }x.
		\]
	\end{teor}
%%%%%%%%%%%%%%%%%%%%%%%%%%%%%%%%%%%%%%%%%
%
%LEZIONE 04/04/2017 - SESTA SETTIMANA (1)
%
%%%%%%%%%%%%%%%%%%%%%%%%%%%%%%%%%%%%%%%%%

	\begin{prop}{Calcolo dei convergenti}{calcoloConvergenti}
	Sia \(x=[a_1;a_2,\ldots,a_n]\in \Q\). Detto \(C_k=\frac{p_k}{q_k}\) il \(k\)-esimo convergente di \(x\), con \((p_k,q_k)=1\), avremo
		\[
		\begin{pmatrix}a_1 & 1\\1 & 0\end{pmatrix}\begin{pmatrix}a_2 & 1\\1 & 0\end{pmatrix} \cdot\ldots\cdot \begin{pmatrix}a_k & 1\\1 & 0\end{pmatrix} = \begin{pmatrix}p_k & p_{k-1}\\q_k & q_{k-1}\end{pmatrix}
		\]
	\end{prop}

	Passiamo ora all'attacco vero e proprio. Ricordiamo che per ipotesi
		\[
		\sqrt{6}d < \sqrt[4]{N} \qquad\text{e}\qquad q < p < 2q.
		\]
	In quanto attaccanti conosciamo solo la chiave pubblica \((N,e)\). Inoltre sappiamo che
		\[
		e\,d \equiv 1 \pmod{\j(N)} \implies e\,d = 1 + k\,\j(N).
		\]
	Vogliamo dimostrare che sotto le nostre ipotesi \(\frac{k}{d}\) è un convergente di \(\frac{e}{N}\).
	Per farlo sfruttiamo il teorema sui convergenti enunciato in precedenza. Dobbiamo quindi dimostrare che
		\[
		\abs*{\frac{e}{N} - \frac{k}{d}} < \frac{1}{2d^2}
		\]
	Procediamo con la stima:
		\[
		\abs*{\frac{e}{N} - \frac{k}{d}} = \frac{\abs{e\,d-k\,N}}{N\,d} = \frac{\abs{e\,d-k\,\j(N)+k\,\j(N)-k\,N}}{N\,d},
		\]
	Da \(e\,d = 1 + k\,\j(N)\) segue \(e\,d-k\,\j(N)=1\), per cui
		\[
		\begin{split}
		\abs*{\frac{e}{N} - \frac{k}{d}} & = \frac{\abs{e\,d-k\,\j(N)+k\,\j(N)-k\,N}}{N\,d} = \frac{\abs{1+k\,\big(\j(N)-N\big)}}{N\,d} = \frac{\abs{-1+k\,\big(N-\j(N)\big)}}{N\,d}\\
		& < \frac{k\,\abs{N-\j(N)}}{N\,d} = \frac{k\,\big(N-\j(N)\big)}{N\,d},
		\end{split}
		\]
	ora
		\[
		p<2q \implies N-\j(N) = p\,q-(p-1)(q-1) = p+q-1 < 3q-1 < 3q,
		\]
	da cui
		\[
		\abs*{\frac{e}{N} - \frac{k}{d}} < \frac{K\,\big(N-\j(N)\big)}{N\,d} < \frac{k\,3q}{N\,d};
		\]
	inoltre
		\[
		q<p \implies N=p\,q > q^2 \implies q < \sqrt{N},
		\]
	quindi
		\[
		\abs*{\frac{e}{N} - \frac{k}{d}} < \frac{k\,3q}{N\,d} < \frac{3k\,\sqrt{N}}{N\,d} = \frac{3k}{d\,\sqrt{N}}.
		\]
	Infine \(e\,d-k\,\j(N)=1\) ci dice che \(e<\j(N) \implies k<d\), quindi
		\[
		\abs*{\frac{e}{N} - \frac{k}{d}} < \frac{3k}{d\,\sqrt{N}} < \frac{3d}{d\,\sqrt{N}} = \frac{3}{\sqrt{N}}.
		\]
	Ricordando che, per ipotesi, \(\sqrt{6}d < \sqrt[4]{N}\), avremo \(6d^2 < \sqrt{N}\). Quindi
		\[
		\abs*{\frac{e}{N} - \frac{k}{d}} < \frac{3}{\sqrt{N}} < \frac{3}{6 d^2} = \frac{1}{2d^2}.
		\]
	Quindi, per il teorema precedente, \(\frac{k}{d}\) è un convergente di \(\frac{e}{N}\).
	Possiamo quindi sviluppare \(\frac{e}{N}\) in frazioni continue e controllare tutti i convergenti che hanno la caratteristica della coppia \((k,d)\).

	\begin{oss}
	Il numero di convergenti di \(\frac{n}{m}\) è pari al numeri di passi dell'algoritmo euclideo, il quale è minore della lunghezza massima fra \(n\) e \(m\).
	I convergenti sono pertanto in numero di \(\bO\big(L(N)\big)\).
	\end{oss}
	\noindent
	Per trovare il convergente corretto dobbiamo cercare una coppia \((k,d)\) che sia plausibile.
	Non potremmo quindi accettare, ad esempio, \(d\) pari.
	Tramite la conoscenza di \(N,e\) calcoliamo
		\[
		\j(N) = \frac{ed-1}{k},
		\]
	che deve essere intero.
	Tramite \(N\) e il possibile \(\j(N)\) possiamo fattorizzare \(N\) e verificare quindi se la coppia \((k,d)\) è corretta.
	Sappiamo infatti che
		\[
		\begin{cases}
		p\,q = N\\
		p+q = N-\j(N)+1
		\end{cases}
		\]
	quindi \(p,q\) si trovano risolvendo l'equazione
		\[
		x^2-(p+q)\,x + p\,q.
		\]

	\begin{ese}
	Supponiamo che la chiave pubblica sia \(N=834443\) e \(e=499565\).
	Affinché l'attacco abbia successo si deve avere \(\sqrt{6}d < \sqrt[4]{N}\). Ora
		\[
		30 < \sqrt[4]{N} < 31,
		\]
	quindi non considereremo eventuali \(d>31\) nel calcolo dei convergenti.
	Sviluppiamo \(\frac{e}{N}\) in frazione continua e calcoliamone i convergenti:
		\begin{align*}
		499565 = 0 \cdot 834443 + 499565 & \implies C_1 = 0\\
		834443 = 1 \cdot 499565 + 334878 & \implies C_2 = 1\\
		499565 = 1 \cdot 334878 + 164685 & \implies C_3 = \frac{1}{2}\\
		334878 = 2 \cdot 164685 + 5504 & \implies C_4 = \frac{3}{5}\\
		164685 = 29 \cdot 5504 + 5071 & \implies C_5 = \frac{89}{147} 
		\end{align*}
	Da \(C_5\) in poi i convergenti non sono più accettabili in quanto avrebbero \(d>31\).
	Inoltre \(C_1,C_2\) sono chiaramente inaccettabili e \(C_2\) non è accettabile poiché \(d\) è necessariamente dispari.
	L'unica possibilità si ha quindi con la coppia \(k=3,d=5\). In particolare
		\[
		\j(N) = \frac{e\,d-1}{k} = 832608
		\]
	che è plausibile.
	Impostiamo la seguente equazione di secondo grado per il calcolo di \(p,q\):
		\[
		x^2-(p+q)\,x+p\,q = 0 \iff x^2 - 1836 x + 834443 = 0
		\]
	da cui
		\[
		p,q = \frac{1836\pm\sqrt{1836^2-4 \cdot 834443}}{2} = \frac{1836\pm 182}{2} \implies
		\begin{cases}
		p = 1009\\
		q = 827
		\end{cases}
		\]
	Che verificano la fattorizzazione di \(N\).
	\end{ese}

\subsection{Broadcast attack}

	Nel paragrafo precedente abbiamo visto come una scelta particolare dell'esponente di decifratura, renda possibile un attacco ad RSA.
	In questo paragrafo vedremo che lo stesso accade per scelte dell'esponente di cifratura molto piccole.
	Questo attacco è particolarmente interessante poiché nelle prime implementazioni di RSA si sceglieva sempre \(e=3\).
	Questo approccio, ora superato, presenta un evidente problema nella cifratura
		\[
		x \longmapsto x^3 \pmod{N}
		\]
	infatti, se \(x\) è sufficientemente piccolo, si può avere che \(x^3<N\) per cui non vi è nessuna riduzione modulare e il messaggio può essere estratto tramite una semplice radice cubica.

	D'altronde un esponente di cifratura molto piccolo espone il crittosistema ad un vero e proprio attacco, detto \emph{broadcast attack}.
	Supponiamo di voler mandare lo stesso messaggio a \(k\) utenti distinti con moduli RSA \(N_1,\ldots,N_k\) diversi ma con lo stesso esponente di cifratura \(e_1= \ldots = e_k = k\).
	Otterremo quindi \(k\) testi cifrati:
		\[
		\begin{cases}
		x^k \equiv y_1 \pmod{N_1}\\
		x^k \equiv y_2 \pmod{N_2}\\
		\vdots\\
		x^k \equiv y_k \pmod{N_k}
		\end{cases}
		\]
	Da attaccanti siamo interessati a studiare il sistema
		\[
		\begin{cases}
		z \equiv y_1 \pmod{N_1}\\
		z \equiv y_2 \pmod{N_2}\\
		\vdots\\
		z \equiv y_k \pmod{N_k}
		\end{cases}		
		\]
	Per il teorema cinese di resti, vi è un'unica soluzioni modulo \(N_1 \cdot\ldots\cdot N_k\) e tale soluzione è proprio \(x^k\) come numero intero. Infatti
		\[
		x < N_1, \ldots, N_k \implies x^k < N_1 \cdot\ldots\cdot N_k.
		\]
	Pertanto possiamo trovare il messaggio \(x\) estraendo la \(k\)-esima radice da \(x^k\).

	Chiaramente questo attacco ha successo tanto più è piccolo l'esponente di cifratura \(k\).

	\begin{ese}
	Supponiamo che tre utenti \(A_1,A_2,A_3\) ricevano lo stesso messaggio \(m=1500\) e supponiamo che le chiavi pubbliche degli utenti siano rispettivamente
		\[
		(1711,3); \qquad (1643,3); \qquad (1739,3).
		\]
	Da attaccante intercettiamo i tre messaggi cifrati
		\[
		y_1 = 1170; \qquad y_2 = 333; \qquad y_3 = 970.
		\]
	Possiamo quindi impostare il sistema
		\[
		\begin{cases}
		x \equiv 1170 \pmod{N_1}\\
		x \equiv 333 \pmod{N_2}\\
		x \equiv 970 \pmod{N_3}
		\end{cases}
		\]
	Il sistema ha un'unica soluzione modulo \(N_1 N_2 N_3\):
		\[
		3375000000 = 1500^3
		\]
	che ci fornisce il messaggio.
	\end{ese}

	Ad oggi la scelta consigliata per \(e\) è \(2^{16}+1=65537\).
	La forma \(2^n+1\) comporta un'applicazione molto semplice dell'algoritmo delle esponenziazioni modulari, così da rendere più efficiente la cifratura.

	\begin{oss}
	Una scelta casuale di \(e\) porta a circa \(1000\) moltiplicazioni durante l'esponenziazione modulare contro le \(2\) della scelta \(2^n+1\).
	\end{oss}

\section{Crittosistema di Rabin}

	Il crittosistema che analizzeremo in questo paragrafo è basato, come nel caso di RSA, sul problema della fattorizzazione. La particolarità che ne rende interessante lo studio è che, a differenza di RSA, la decifratura è del tutto equivalente alla fattorizzazione

	Formalmente il crittosistema di Rabin ha le seguenti caratteristiche:
	\begin{itemize}
		\item \(P=C=\Z_N\) con \(N=p\,q\) dove \(p,q\) sono primi tali che \(p,q\equiv 3 \pmod{4}\).
		\item Lo spazio delle chiavi è dato da
			\[
			K = (N,p,q),
			\]
		dove \(N\) è la chiave pubblica mentre \(p,q\) sono quella privata
		\item Fissata la chiave \(k\) si ha
			\[
			e_k(x) = x^2 \pmod{N} \qquad\text{e}\qquad d_k(y) = \sqrt{y} \pmod{N}
			\]
	\end{itemize}

	Questa funzione di cifratura ha un evidente problema, non è iniettiva.
	Infatti, abbiamo visto in precedenza che, se \(N=p\,q\), la congruenza \(x^2 \equiv y \pmod{N}\) ha zero oppure quattro soluzioni.

	Per decifrare, una volta ricevuto \(y\), per prima cosa si controlla che
		\[
		\lege{y}{p} = \lege{y}{q} = 1,
		\]
	se questo è falso il messaggio è stato cifrato male. Dopo si risolve il sistema
		\[
		\begin{cases}
		x^2 \equiv y \pmod{p}\\
		x^2 \equiv y \pmod{q}
		\end{cases}
		\]
	dal momento che \(p,q\equiv 3 \pmod{4}\) la risoluzione di tali congruenze è particolarmente semplice. Per tale proprietà è lecito calcolare
		\[
		z_p = y^{\frac{p+1}{4}}\pmod{p} \qquad\text{e}\qquad z_q = y^{\frac{q+1}{4}} \pmod{q}.
		\]
	Ora
		\[
		z_p^2 = y^{\frac{p+1}{2}} = y\,y^{\frac{p-1}{2}} \equiv y \pmod{p}
		\]
	e analogamente \(z_q^2 \equiv y \pmod{q}\). Per cui ho trovato le quattro radici \(\pm z_p\) e \(\pm z_q\). A questo punto per trovare le soluzioni complessive del sistema devo risolvere quattro sistemi di congruenze:
		\[
		\begin{cases}
		w \equiv \pm z_p \pmod{p}\\
		w \equiv \pm z_q \pmod{q}
		\end{cases}
		\qquad\text{e}\qquad
		\begin{cases}
		w \equiv \pm z_p \pmod{p}\\
		w \equiv \mp z_q \pmod{q}
		\end{cases}
		\]
	Anche in questo caso vi è un modo più immediato per il calcolo delle soluzioni.
	Calcolando l'identità di Bezout \(1=s\,p+t\,q\), si ottengono le \(4\) soluzioni tramite
		\[
		\pm w_1 = \pm(s\,p\,z_q + t\,q\,z_p) \qquad\text{e}\qquad \pm w_2 = \pm(s\,p\,z_q - t\,q\,z_p).
		\]
	Per ovviare al problema che ad ogni testo cifrato corrispondono quattro testi in chiaro si utilizzano delle tecniche di \emph{ridondanza}. Di fatto si copia una porzione del messaggio e la si ripete in fondo allo stesso, così da rendere evidente quale testo in chiaro è quello corretto.
	Si può ad esempio decidere, per messaggi di \(k\) bit, di inserire il messaggio nei primi \(2/3\) bit, mentre l'ultima parte del testo conterrà una ripetizione degli ultimi bit del messaggio.

	\begin{ese}
	Sia \(N=84281=271 \cdot 311\). Poiché \(L(N)=17\) il nostro messaggio dovrà essere di \(16\) bit. Scegliamo quindi di inserire il nostro testo nei primi \(10\) bit e di ripeterlo negli ultimi \(6\).
	Quindi per inviare \(m=1001111001_2\) dovremo cifrare il messaggio
		\[
		\bar{m} = {1001111001\underbrace{111001}}_2
		\]
	che in decimale corrisponde ad \(x=40569\). Il testo cifrato sarà pertanto
		\[
		y = x^2  \equiv 4393 \pmod{N}.
		\]
	Per decifrare, una volta verificato che
		\[
		\lege{4393}{271} = \lege{4393}{311} =1,
		\]
	si deve risolvere il sistema
		\[
		\begin{cases}
		z^2 \equiv 4393 \pmod{271}\\
		z^2 \equiv 4393 \pmod{311}
		\end{cases}
		\]
	che come sappiamo si ottengono da
		\[
		z_p = 4393^{68} \equiv 81 \pmod{p} \qquad\text{e}\qquad z_q = 4393^{78} \equiv 138 \pmod{311}
		\]
	Tramite l'identità di Bezout \(1=-70 \cdot 271 + 61 \cdot 311\), troviamo 
		\[
		w_1 = 61 \cdot 311 \cdot 81 - 70 \cdot 271 \cdot 138 \qquad\text{e} w_2 = 61 \cdot 311 \cdot 81 + 70 \cdot 271 \cdot 138
		\]
	che ci forniscono le quattro soluzioni del sistema
		\[
		79755; \qquad 4526; \qquad 40569; \qquad 43712.
		\]
	Passando in binario si osserva immediatamente che solamente \(40569\) esibisce una ridondanza, ed è pertanto il messaggio corretto.
	\end{ese}

	Come già accennato, l'interesse teorico  del crittosistema di Rabin deriva dal fatto che attaccarlo con successo è del tutto equivalente alla fattorizzazione di \(N\).
	Chiaramente, come per RSA, se conosco la fattorizzazione posso decifrare, dobbiamo quindi mostrare il viceversa.
	Supponiamo quindi di saper decifrare un testo cifrato di Rabin, questo significa che sono in grado di estrarre una radice quadrata modulo \(N\).
	Prendiamo quindi un \(r\) e calcoliamo \(y=r^2 \pmod{N}\). Adesso sfruttiamo la nostra capacità di estrarre una radice quadrata. Se otteniamo \(\pm r\) non possiamo ottenere ulteriori informazioni e dobbiamo cambiare base; se invece otteniamo una delle altre due radici \(\pm s\) avrò la seguente informazione
		\[
		r \not\equiv s \pmod{N} \qquad\text{ma}\qquad r^2 \equiv s^2 \pmod{N}.
		\]
	Pertanto
		\[
		r^2 - s^2 \equiv 0 \pmod{N} \implies (r-s)(r+s) \equiv 0 \pmod{N} \implies (r-s)(r+s) = k\,N.
		\]
	Per cui, calcolando \((r+s,N)\) o \((r-s,N)\), otteniamo la fattorizzazione di \(N\).
%%%%%%%%%%%%%%%%%%%%%%%%%%%%%%%%%%%%%%%%%%%
%
%LEZIONE 20/04/2017 - SETTIMA SETTIMANA (1)
%
%%%%%%%%%%%%%%%%%%%%%%%%%%%%%%%%%%%%%%%%%%%
\section{Algoritmi di fattorizzazione}

	Un algoritmo di fattorizzazione considera tipicamente interi \(N\) dispari che non siano una potenza perfetta\graffito{\(N\neq a^k\)}.
	Lo scopo di tali algoritmi, in generale, non è quello di fattorizzare completamente \(N\), bensì di spezzarlo in maniera non banale, ovvero di scrivere
		\[
		N = n_1 n2 \qquad\text{con }1 < n_1,n_2 < N.
		\]
	Nel caso del modulo RSA di fatto spezzare è equivalente a fattorizzare \(N\).

	Distinguiamo due tipi di algoritmi di fattorizzazione: "speciali" e "generali".
	I primi operano quando \(N\) soddisfa determinate ipotesi e saranno, in questi particolari casi, molto efficienti; i secondi si applicano ad \(N\) qualsiasi. Nella nostra trattazione descriveremo perlopiù algoritmi speciali.

	\subsection{Algoritmo p-1 di Pollard}

	La descrizione di questo algoritmo avverrà supponendo che \(N=p\,q\).
	Ipotizziamo di conoscere un intero \(L\) tale che
		\[
		p-1 \mid L.
		\]
	Scegliamo quindi \(a<N\) e assumiamo che \((a,N)\neq 1\), altrimenti abbiamo già spezzato \(N\). Ora \(p-1 \mid L \implies L=k\,(p-1)\), quindi
		\[
		a^L = {(a^{p-1})}^k \equiv 1 \pmod{p} \implies p \mid a^L-1.
		\]
	Pertanto \((a^L-1,N)\) ci fornisce \(p\) oppure \(N\).
	Nel primo caso abbiamo spezzato \(N\), altrimenti anche \(q\mid a^L-1\) e non è possibile concludere nulla.

	Dal momento che la conoscenza di \(L\) è solo ipotetica, prenderemo un intero prodotto di molti fattori per massimizzare la probabilità di ottenere un numero con la proprietà cercata.
	La scelta migliore risulta pertanto essere un fattoriale.

	\begin{oss}
	Vedremo infatti che questo algoritmo ha buone probabilità di successo nel caso in cui \(p-1\) abbia molti fattori piccoli
	\end{oss}
	\noindent
	Scegliamo quindi \(L=r!\) e descriviamo formalmente l'algoritmo:

	\begin{algorithmic}[1]
	\Statex
	\Require{\((a,N)=1, B<N\)}
	%\Statex
	\Let{\(r\)}{\(2\)}
	\Let{\(d\)}{\(1\)}
	\While{\(d=1, r<B\)}
		\Let{\(d\)}{\(\big(a^{r!}-1,N\big)\)}
		\If{\(d\neq 1\)}
			\State \Return{\(d\)}
		\EndIf
		\Let{\(r\)}{\(r+1\)}
	\EndWhile	
	\end{algorithmic}

	Osserviamo che non è necessario calcolare \(a^{r!}-1\) ad ogni iterazione. Infatti
		\[
		a^{r!} = a^{(r-1)!\,r} = {\big(a^{(r-1)!}\big)}^r,
		\]
	quindi ad ogni passo devo calcolare una singola esponenziazione.
	Osserviamo inoltre che, in generale
		\[
		(a,N) = \big(a \pmod{N},N\big),
		\]
	per cui l'esponenziazione è modulare.
	Il singolo passo si ricuce pertanto al calcolo di un'esponenziazione modulare e ad un MCD.
	La complessità di ogni iterazione è pertanto polinomiale, mentre quella complessiva dipende dalla relazione di \(B\) con \(N\). In particolare, detto \(k=L(N)\), la complessità sarà
		\[
		B\,\big(L(B)\,k^2+k^3\big),
		\]
	dove il primo addendo è la complessità dell'esponenziazione modulare, mentre il secondo quella dell'MCD.
	Chiaramente se \(B\) è confrontabile con \(p\), la complessità diventa esponenziale e non è più conveniente dei un algoritmo di divisioni successive.

	L'algoritmo avrà successo nell'ipotesi in cui, scritto \(p-1 = q_1^{\a_1} \cdot\ldots\cdot q_k^{\a_k}\), si abbia \(q_i^{\a_i} < B\) per ogni \(i\). In tal caso avremo infatti
		\[
		p-1 \mid B!.
		\]
	Questa condizione rende l'algoritmo del tipo speciale.

	\begin{oss}
	L'algoritmo è chiaramente generale se non si pongono limiti alla grandezza di \(B\).
	\end{oss}

	\begin{ese}
	Consideriamo \(N=10001, a=2, B=10\) e applichiamo l'algoritmo:
	\begin{enumerate}
		\item \(r=2\):
			\[
			a^{2!} = 2^2 = 4 \qquad\text{e}\qquad (4-1,N) = 1.
			\]
		\item \(r=3\):
			\[
			a^{3!} = 4^3 = 64 \qquad\text{e}\qquad (64-1,N) = 1.
			\]
		\item \(r=4\):
			\[
			a^{4!} = 64^2 \equiv 5539 \pmod{N} \qquad\text{e}\qquad (5539-1,N) = 1.
			\]
		\item \(r=5\):
			\[
			a^{5!} = 5539^5 \equiv 7746 \pmod{N} \qquad\text{e}\qquad (7746-1,N) = 1.
			\]
		\item \(r=6\):
			\[
			a^{6!} = 7746^6 \equiv 1169 \pmod{N} \qquad\text{e}\qquad (7746-1,N) = 73.
			\]
	\end{enumerate}
	L'algoritmo restituisce pertanto \(73\) che ci permette di spezzare \(N\):
		\[
		N = 73*137.
		\]
	\end{ese}

	Nella scelta dei moduli RSA bisogna pertanto evitare primi \(p\) tali che \(p-1\) ha solo fattori primi piccoli. Primi adeguati sono i cosiddetti primi \emph{sicuri}.

	\begin{defn}{Primo sicuro}{primoSicuro}
	Un numero primo \(p\) si dice \emph{forte} se è esprimibile nella forma
		\[
		p = 2q+1,
		\]
	dove \(q\) è un altro numero primo.
	\end{defn}

	\begin{notz}
	Il primo \(q\) si dice primo di \emph{Germain}, dalla matematica Sophie Germain.
	\end{notz}

	\subsection{Algoritmo rho di Pollard}

	Sia \(N=p\,q\). Ipotizziamo di conoscere due interi \(x\) e \(x'\) tali che
		\[
		x \not\equiv x' \pmod{N} \qquad\text{e}\qquad x \equiv x' \pmod{p}.
		\]
	Possiamo quindi calcolare \((x-x',N)\) per ottenere \(p\).
	Per trovare \(x,x'\) fisso un dato iniziale \(x_0\) su cui costruisco una successione di interi
		\[
		x_1, x_2, \ldots, x_t \qquad\text{con }x_{i+1} = f(x_i) \pmod{N},
		\]
	dove \(f\colon \Z_n \to \Z_n, x \mapsto x^2+a\), con \(a=1\) tipicamente.
	Per ogni coppia nella successione calcolo \((x_i-x_j,N)\) fino ad ottenere un risultato diverso da \(1\).

	In questa forma l'algoritmo è computazionalmente pesante per via delle numerose coppie presenti in un insieme anche di piccole dimensioni.
	La scelta delle coppie può essere affinate in modo da non renderla del tutto arbitraria.
	Osserviamo che se \(x_i,x_j\) sono una coppia che soddisfa le condizioni dell'algoritmo, allora anche \(x_{i+1},x_{j+1}\) lo è, infatti
		\[
		x_{i+1}-x_{j+1} = x_i^2+1-x_j^2-1 = (x_i-x_j)(x_i+x_j),
		\]
	per cui chiaramente
		\[
		x_i \equiv x_j \pmod{p} \implies x_{i+1} \equiv x_{j+1} \pmod{p}
		\]
	Questo ci dice che la nostra successione \(x_1,\ldots,x_i,\ldots, x_j, \ldots\) si ripete modulo \(p\).

	\begin{oss}
	Il nome \emph{rho} per l'algoritmo deriva proprio da questa caratteristica.
	Se si rappresenta graficamente la successione identificando gli elementi modulo \(p\) si ottiene una figura che ricorda la lettera \(\r\).
	\end{oss}

	Per ridurre il numero di MCD da calcolare si utilizza il \emph{metodo di Floyd}.
	Si definisce una successione di coppie \((x_i,x_j)\) a partire dalla successione e si calcolerà l'MCD solamente su tali coppie.
	La successione è definita, definito un dato iniziale \(x_0\), dalle leggi
		\[
		\begin{cases}
		x_0 = x_0\\
		x_{i+1} = f(x_i) \pmod{N}
		\end{cases}
		\qquad\text{e}\qquad
		\begin{cases}
		y_0 = x_0\\
		y_{i+1} = f\big(f(y_i)\big) \pmod{N}
		\end{cases}
		\]
	I primi elementi della successione saranno pertanto
		\[
		(x_0,x_0), (x_1,x_2), (x_2,x_4), \ldots
		\]
	Possiamo quindi scrivere formalmente l'algoritmo, considerando \(x_0=2\).

	\begin{algorithmic}[1]
	\Statex
	\Let{\(x\)}{\(2\)};
	\Let{\(y\)}{\(2\)};
	\Let{\(d\)}{\(1\)}
	\While{\(d=1\)}
		\Let{\(x\)}{\(f(x)\)};
		\Let{\(y\)}{\(f\big(f(x)\big)\)}
		\Let{\(d\)}{\((x-y,N)\)}
	\EndWhile
	\If{\(d\neq N\)}
		\State \Return{\(N\)}
	\Else
		\State \Return{Fallimento}
	\EndIf
	\end{algorithmic}

	Osserviamo che nell'algoritmo appena proposto non abbiamo effettivamente calcolato nessuna delle due successioni descritte in precedenza. Bensì abbiamo considerato un elemento alla volta di tali successioni procedendo iterativamente al calcolo dell'MCD.

	\begin{oss}
	Siano \(t,s\) sono gli indici della prima collisione, ovvero tali che
		\[
		x_t \equiv x_s \pmod{p}.
		\]
	Detta \(l=s-t\) l'ampiezza del ciclo di ripetizione all'interno della successione, l'algoritmo di Floyd trova una collisione in al più \(t+l\) passi.
	\end{oss}

	\begin{ese}
	Consideriamo \(N=55, x_0=3, f(x)=x^2+1 \pmod{N}\). Applichiamo l'algoritmo calcolando esplicitamente la successione e la successione di coppie. I primi dieci elementi della successione saranno
		\[
		10,46,27,15,6,37,50,26,17.
		\]
	La successione di coppie sarà pertanto
		\[
		(3,3),(10,46),(46,15),(27,37),(15,26).
		\]
	Calcolando i vari MCD si trova
		\[
		(27-37,N) = 5.
		\]
	Osserviamo che passando modulo \(5\) si individua immediatamente una ripetizione all'interno della successione.
	\end{ese}

	Non vi è una stima precisa della complessità computazionale dell'algoritmo rho. Una eventuale stima si baserebbe sul medesimo concetto del paradosso del compleanno, portando a dedurre che se \(t\ge \sqrt[4]{N}\), la probabilità di successo è maggiore di \(\frac{1}{2}\).

	\subsection{Metodo di Fermat}

	Sia \(N=p\,q\). Ipotizziamo di conoscere \(x,y\in\Z\) tali che
		\[
		N+y^2 = x^2.
		\]
	Allora
		\[
		N= x^2-y^2 = (x-y)(x+y),
		\]
	per cui se \(x-y \neq 1\) abbiamo spezzato \(N\).
	L'algoritmo di Fermat impone un limite superiore \(B\) e considera \(y=1,2,\ldots,B\). Successivamente calcola \(N+y^2\) verificando se è un quadrato perfetto, in tal caso riporta la fattorizzazione. Formalmente

	\begin{algorithmic}[1]
	\Statex
	\Require{B<N}
	\Let{\(y\)}{\(1\)}
	\Let{\(x^2\)}{\(N+y^2\)}
	\While{\(x^2 \neq \square\)}
		\Let{\(y\)}{\(y+1\)}
		\Let{\(x^2\)}{\(N+y^2\)}
	\EndWhile
	\State \Return{\(x-y\)}
	\end{algorithmic}

	Se non viene imposto un limite \(B\), l'algoritmo termina per al più
		\[
		y = \frac{N-1}{2} \qquad\text{infatti } N + \left( \frac{N-1}{2} \right)^2 =\left( \frac{N+1}{2} \right)^2
		\]
	Questo algoritmo è efficiente se \(N=p\,q\) e \(\abs{p-q}\) è piccolo.
%%%%%%%%%%%%%%%%%%%%%%%%%%%%%%%%%%%%%%%%%%
%
%LEZIONE 27/04/2017 - OTTAVA SETTIMANA (1)
%
%%%%%%%%%%%%%%%%%%%%%%%%%%%%%%%%%%%%%%%%%%
	
	\subsection{Metodo delle basi di fattorizzazione}

	I moderni algoritmi di fattorizzazione si basano sull'idea di applicare il metodo di Fermat modulo \(N\).
	Si cerca quindi di trovare \(x,y\) tali che
		\[
		x \not\equiv \pm y \pmod{N} \qquad\text{e}\qquad x^2 \equiv y^2 \pmod{N}.
		\]
	In tal caso \(N\mid x^2-y^2=(x-y)(x+y)\), per cui si può scomporre \(N\) calcolando
		\[
		(N,x+y) \qquad\text{e}\qquad (N,x-y).
		\]
	Una tecnica per la ricerca di tali \(x,y\) si chiama delle \emph{basi di fattorizzazione}.

	\begin{defn}{Interi B-smooth}{interiBSmooth}\index{B-smooth}
	Un intero \(n\) si dice \(B\)-smooth se nessuno dei suoi fattori primi è maggiore di \(B\). Ovvero
		\[
		n = p_1^{\a_1} \cdot\ldots\cdot p_s^{\a_s} \implies p_i \le B \,\fa i.
		\]
	\end{defn}

	\begin{notz}
	Un intero \(B\)-smooth si dice anche \(B\)-numero.
	\end{notz}

	\begin{ese}
	\(n=504\) è \(7\)-smooth, infatti \(504=2^3 \cdot 3^2 \cdot 7\).
	\end{ese}

	\begin{defn}{B-vettore}{BVettore}\index{B-vettore}
	Supponiamo che \(n\) sia un \(B\)-numero. Scriviamo la fattorizzazione di \(n\) facendo comparire tutti i primi minori o uguali a \(B\) con le rispettive potenze, supponendo che vi siano \(b\) primi minori di \(B\), avremo
		\[
		n = p_1^{\a_1} \cdot p_2^{\a_2} \cdot\ldots\cdot p_b^{\a_b} \qquad\text{con }\a_i \ge 0.
		\]
	Possiamo quindi associare ad \(n\) un \(B\)-vettore composto dai \(b\) esponenti della precedente fattorizzazione, ovvero
		\[
		v_n = (\a_1, \a_2, \ldots, \a_b).
		\]
	\end{defn}

	\begin{ese}
	Consideriamo l'intero dell'esempio precedente: \(n=2^3 \cdot 3^2 \cdot 7\). \(n\) è \(7\)-smooth, per cui possiamo associargli un \(7\)-vettore scrivendone la sua fattorizzazione completa:
		\[
		n = 2^3 \cdot 3^2 \cdot 5^0 \cdot 7 \implies v_n = (3,2,0,1).
		\]
	Chiaramente la scelta di \(7\) per descrivere \(n\) è arbitraria, infatti \(n\) è \(B\)-smooth per ogni \(B\ge 7\). Per \(B>7\) dovremmo considerare anche i primi \(7<p\le B\), ad esempio se \(B=12\) avremo
		\[
		n = 2^3 \cdot 3^2 \cdot 5^0 \cdot 7 \cdot 11^0 \implies v_n =  (3,2,0,1,0).
		\]
	\end{ese}
	\noindent
	L'idea delle basi di fattorizzazione è quella di trovare vari interi \(z\) tali che \(z^2 \pmod{N}\) sia \(B\)-smooth rispetto ad un'appropriata scelta di \(B\).

	Nel prossimo esempio mostreremo il procedimento per sfruttare la conoscenza di tali \(z\), giustificandone in seguito la scelta

	\begin{ese}
	Consideriamo \(N=84923\) e supponiamo di voler cercare interi \(7\)-smooth.
	Come detto supponiamo di aver trovato, giustificandolo in seguito, i seguenti interi \(z_1=513\) e \(z_2=537\). Ora
		\[
		z_1^2 \pmod{N} = 8400 = 2^4 \cdot 3 \cdot 5^2 \cdot 7 \qquad\text{e}\qquad z_2^2 \pmod{N} = 33600 = 2^6 \cdot 3 \cdot 5^2 \cdot 7
		\]
	quindi entrambi sono, modulo \(N\), interi \(7\)-smooth. Consideriamo ora
		\[
		x = z_1 \cdot z_2 \pmod{N} = 20712 \qquad\text{e}\qquad y^2 = z_1^2 \cdot z_2^2,
		\]
	dove \(y^2\) è un quadrato perfetto in quanto
		\[
		z_1^2 \cdot z_2^2 = 2^{10} \cdot 3^2 \cdot 5^4 \cdot 7^2 = {(2^5 \cdot 3 \cdot 5^2 \cdot 7)}^2 = 16800^2.
		\]
	Abbiamo quindi trovato \(x,y\) tali che \(x\not\equiv \pm y\pmod{N}\) ma \(x^2 \equiv y^2 \pmod{N}\). Possiamo quindi fattorizzare \(N\) calcolando
		\[
		(N,20712-16800) = 163 \qquad\text{e}\qquad (N,20712+16800) = 521.
		\]
	\end{ese}
	\noindent
	Osserviamo che nell'esempio \(z_1,z_2\) sono tali che
		\[
		v_{z_1} + v_{z_2} = (4,1,2,1)+(6,3,2,1) = (10,2,4,2) \equiv (0,0,0,0) \pmod{2}.
		\]
	Stiamo quindi cercando una relazione di dipendenza lineare sullo spazio vettoriale \(\Z_2^4\).

	In generale, avremo che vi sono \(b\) primi minori o uguali di \(B\), quindi i \(B\)-vettori modulo \(2\) sono vettori dello spazio vettoriale \(\Z_2^b\).
	Se scelgo \(z_1,\ldots,z_c\) tali che \(z_i^2\) modulo \(N\) è un \(B\)-numero, affinché si abbiamo sicuramente una relazione del tipo
		\[
		a_1 v_{z_1} + a_2 v_{z_2} + \ldots + a_c v_{z_c} = (0,0, \ldots, 0),
		\]
	si deve avere \(c \ge b+1\), poiché, come sappiamo, \(b+1\) vettori sono dipendenti in uno spazio vettoriale di dimensione \(b\).

	L'algoritmo che ci apprestiamo a discutere, detto \emph{Dixon's random square}, sceglie in maniera casuale, ma opportuna, i possibili \(z_i\).
	Di fatto considera
		\[
		z_{i,k} = \lfloor\sqrt{k\,N}\rfloor + i \qquad\text{con }k = 1,\ldots,l_k \text{ e }i = 1,\ldots,l_i.
		\]
	Questa scelta fa in modo che \(z_{i,k}^2\) sia piccolo modulo \(N\) e che, pertanto, il calcolo della sua fattorizzazione sia facile.

	\begin{ese}
	Consideriamo \(N=32033\) e applichiamo l'algoritmo per trovare dei possibili \(z_i\):
		\[
		\begin{array}{c|c|c|c}
		\toprule
		k & x=\lfloor\sqrt{k\,N}\rfloor+1 & x^2 \pmod{N} & \text{Fattorizzazione di }x^2\\
		\midrule
		1 & 179 & 8 & 2^3\\
		2 & 254 & 450 & 2 \cdot 3^2 \cdot 5^2\\
		3 & 310 & 1 & 1\\
		4 & 358 & 32 & 2^5\\
		\bottomrule
		\end{array}
		\]
	fingiamo di non aver trovato il terzo elemento che corrisponde ad una radice non banale dell'unità e che ci permetterebbe una facile fattorizzazione.
	Consideriamo \(z_1\) e \(z_4\) i cui quadrati sono \(2\)-smooth e il cui prodotto costituisce un quadrato perfetto, avremo quindi
		\[
		x = z_1 \cdot z_4 \pmod{N} = 16 \qquad\text{e}\qquad y^2 = 2^8 \implies y = 16,
		\]
	questa scelta non è quindi accettabile in quanto \(x \equiv y \pmod{N}\).
	Consideriamo quindi \(z_1\) e \(z_2\) i cui quadrati sono \(5\)-smooth. Avremo
		\[
		x = z_1 \cdot z_2 \pmod{N} = 13433 \qquad\text{e}\qquad y^2 = 2^4 \cdot 3^2 \cdot 5^2 \implies y = 60.
		\]
	Possiamo quindi fattorizzare calcolando
		\[
		(N,13433-60) = 311 \qquad\text{e}\qquad (N,13433+60) = 103.
		\]
	\end{ese}

	La complessità di questo algoritmo si stima essere sotto-esponenziale, in questo corso non ci addentreremo nel calcolo.